\documentclass[a5paper,6pt]{book}
\usepackage[ngerman]{babel} % Deutsch
\usepackage[utf8]{inputenc} % Dateiencodeing - wichtig
\usepackage{graphicx} % Bilder
%\usepackage{yfonts,color} %Initialien
%\usepackage[T1]{fontenc}
\usepackage{calligra}
\usepackage{amssymb}
%\newcommand*\initfamily{\usefont{U}{RoyalIn}{xl}{n}}
\pagestyle{headings}

\begin{document}



\chapter{Das Feld der Ehre}


Dies hier ist das Theaterstück zur Geschichte “Das Feld der Ehre”. Eine Lutz und Jerevan Geschichte adaptiert für die Bühnen Grenzbruecks, Condras und Engoniens vom 
ehrenwerten Theater Haberstaedt.
Es werden durschschnittlich viele Schauspieler und Requisiten benötigt. Der Text ist relativ leicht zu lernen und Interpretationen und spontane Improvisationen 
durchaus
möglich und auch angebracht.

Für Abendvorstellungen kann das Stück an mehreren Stellen zotiger gestaltet werden, aber an eine wirkliche Komödie wird es nie herankommen. Soweit möglich haben wir
Erfahrungen einfließen lassen und Möglichkeiten zur Abänderung oder für Einschübe markiert.

Das Stück besitzt keine Pause und kann ohne Umbauten oder Szenenunterbrechungen durchgespielt werden.



\newpage



\section{Das Stück}

\twocolumn
\setlength{\columnseprule}{0,1mm}

\textbf{Lutz}: Aber Herr, ich verstehe immer noch nicht warum wir in der kommenden Schlacht nicht auf Seiten der Tiburer kämpfen. Man sagt sie haben die besten Reiter überhaupt.

\textbf{Jerevan}: Weil diese ach so noblen Tiburer niemals einen dahergelaufenen fahrenden Ritter wie mich in ihren Reihen dulden würden, der kaum mehr besitzt als das, was er am Leibe trägt.

\textbf{Lutz}: Aber Herr eure Künste mit dem Schwert sind legendär und suchen landein, landaus ihres Gleichen.

\textbf{Jerevan}: In der Schlacht zählt das Können des Einzelnen kaum und wenn diese Ritter mit ihren Ahnentafeln, so lang, wie mein Arm nicht mit dir kämpfen wollen, weil dein Blut nicht so blau ist, wie ihres, dann ist die Schlacht schon für dich vorbei, bevor die Armeen das Feld der Ehre betreten haben.

\textbf{Lutz}: Aber Herr, jeder Ritter, der die Tugenden befolgt soll doch ein Bruder des Schwertes und in ihrer Gemeinschaft gleich sein.

\textbf{Jerevan}: Habe ich dir denn gar nichts beigebracht? Ein Ritter ist nichts anderes, als ein Kämpfer mit Waffe und Rüstung und der Absicht zu töten. Nicht anderes, als ein Mörder, der die ganzen Dinge, wie Ruhm, Ehre und die Gunstbezeugungen der Damen darüberstreut um vor anderen und sich selber zu verbergen, was er wirklich ist.

\textbf{Lutz}: Aber Ritter beschützen doch die Schwachen, wie diese Bauern dort.

\begin{quote}
  \textit{- Auftritt Margaret und ihre Zofe in zu sauberer Bauernkleidung, die ihre hochherrschaftlichen Gewänder nur notdürftig verdeckt. Die Zofe mit einem Korb am Arm.-
}
 \end{quote}

\textbf{Jerevan}: Zwei Bauersfrauen allein auf dem Weg zum Schlachtfeld? Habt ihr keine Angst, dass ihr überfallen werden könntet, … von einem Raubritter, oder sonst einem fiesen Schergen?

\textbf{Margaret}: Aber meine Tochter und ich sind doch nur zwei arme Bäuerinnen auf der Suche nach unserer Kuh. Wer sollte uns denn etwas antun wollen?

\textbf{Lutz}: Ihr braucht keine Angst zu haben. Herr Jerevan von Eibenhain ist ein ehrenwerter Ritter, der euch sicher beschützen wird bei eurer Aufgabe.

\textbf{Jerevan}: Wird er?

\textbf{Lutz}: Ja, die beiden Damen werden sicher dankbar sein für die Hilfe.

\textbf{Margaret}: Ja, sehr dankbar.

\textbf{Jerevan}: Ah! Es ist mir immer wieder eine Freude, wenn die Fundamente unserer Gesellschaft sich als dankbar erweisen wolle, … da sie meist in Naturalien bezahlen.
Ja, ich gelobe euch bei eurer Aufgabe zu helfen.

\textbf{Lutz}: Ja Herr, das sollten wir …

\textbf{Jerevan}: Ach Quatsch. Du wirst jetzt erst mal ein Lager machen und hier mit der Tochter Holz sammeln. Die gute Frau und ich gehen so lange mal ein Stück spazieren.

\textbf{Lutz}: Ja, aber Herr …

\textbf{Jerevan}: Nix “aber Herr”, tu was ich gesagt hab, oder du kriegst noch nen Satz hinter die Ohren.

\begin{quote}
  \textit{- Abgang Jerevan und Margaret - Lutz und Zofe bleiben alleine zurück und sammeln Holz. Die Zofe hebt sehr anmutig immer kleine Ästchen auf, die mal eigentlich nicht benutzen kann.
Bei der Abendvorstellung beugt sie sich immer mit dem verlängerten Rücken zum Publikum.}
 \end{quote}

\textbf{Lutz}: Bauern, ja?

\textbf{Zofe}: Ja, wir sind zwei einfache Bäuerinnen auf der Suche nach unserer Kuh.

\textbf{Lutz}: Ja, ähem, natürlich, aber ihr solltet wirklich aufpassen ein Schlachtfeld ist kein Platz für allein reinsende … Bauersfrauen. Es ist gefährlich.

\textbf{Zofe}: Auch wenn dies hier ein Schlachtfeld ist, so gibt es doch viele edle Herren hier, die der hohen .. meiner Mutter nichts antun würden.

\textbf{Lutz}: Bei meinem Herren Jerevan braucht ihr euch nicht zu fürchten. Es mag ungehobelt und schroff sein, aber in seinem Kern ist er ein guter Mann und sein Wort würde er niemals brechen.
Andere Ritter wären da sicher nicht so. Ich fürchte der Krieg bringt das Schlechteste in den Menschen hervor und es gäbe sicher einige Ritter, die eure Zwangslage ausnutzen würden zu ihrem persönlichen Vorteil.

\textbf{Zofe}: Doch gerade in diesen wiedrigen Zeiten kommen der wahre Edelmut und die Tugenden zum Vorschein, der edles Blut vom tumben Volk unterscheidet…

\textbf{Lutz}: Das edle Blut hat noch niemanden davor bewahrt Menschen auszunutzen. Im Gegenteil, die schlimmsten Dinge wurden im Namen angeblich edlen Blutes getan.
Die Tugenden der Ritter sind es, die ihn erheben und zu etwas besserem machen, nicht sein Blut.

\textbf{Zofe}: Die einzige Tugend, die jene von uns, die nicht von edlem Stand sind, beachten sollten ist die Demut.

\begin{quote}
  \textit{(weiter Stöcke sammelnd)}
 \end{quote}


\textbf{Lutz}: Was außer Demut noch wichtig ist, das ist die Wahrheit. Ich kann euch helfen, wirklich, wenn ihr mir die Wahrheit sagt.

\textbf{Zofe}: Schwöre zunächst, dass du nichts von dem, was ich dir jetzt sagen werde weiter erzählst.

\textbf{Lutz}: Ich schwöre es.

\textbf{Zofe}: Ich diene ihrer Hoheit Margarethe von Hohenstaaden.

\textbf{Lutz}: Was?!

\textbf{Zofe}: Ich habe wirklich alles versucht, aber die hohe Herrin ließ sich durch nichts umstimmen; sie will diese Aufgabe unbedingt selbst in die Hand nehmen.

\textbf{Lutz}: Was für eine Aufgabe?

\textbf{Zofe}: Das Land leidet unter diesem Krieg. Die Bauern flehen Myrn um Frieden an. Die hohe Herrin hat sich aufgemacht bei ihrem Herren Gemahl, vor den Augen aller seiner Edlen, Frieden zu erbitten. Er wird es ihr nichts abschlagen können, nicht hier.

\textbf{Lutz}: Ich werde alles tun, um dir zu helfen. Aber leise jetzt, hier kommen mein Herr und deine Herrin.

\begin{quote}
  \textit{- Auftritt Jerevan und Magdalena -}
 \end{quote}


\textbf{Jerevan}: Was? Das soll ein Lager sein?

\textbf{Lutz}: Ah, Herr … Ähem, seid ihr weit gekommen?

\begin{quote}
  \textit{- zieht Margaret vom Ritter weg und positioniert sich zwischen ihnen -}
 \end{quote}

\textbf{Jerevan}: Hä? … hm … heute werden wir nicht weiter gehen. Auf dem Schlachtfeld liegen nur noch die Überbleibsel des letzten Tages und die Heere haben sich noch nicht aufgestellt. Wir werden morgen, bevor es hell wird, zu den Hohenstaadenern gehen.
Lutz, Wein!

\textbf{Lutz}: Das geht nicht Herr, den habt ihr gestern aufgetrunken.

\begin{quote}
  \textit{- Auftritt Bastard -}
 \end{quote}



\textbf{Bastard}: Dann trifft es sich ja gut, dass ich noch welchen habe.

\begin{quote}
  \textit{- Jerevan springt auf, geift ans Schwert, zieht aber noch nicht -}
 \end{quote}

\textbf{Jerevan}: Bleib stehen, wer bist du?

\textbf{Bastard}: Der Bastard von Malmais, ein müder Ritter, der ein warmes Feuer gegen einen Becher Wein tauscht, wenn ihr auf der richtigen Seite steht.
Tibur oder Hohenstaaden?

\textbf{Jerevan}: Hohenstaaden, ab dem morgigen Tage.


\begin{quote}
  \textit{- Bastard steckt sein Schwert weg -}
 \end{quote}


\textbf{Bastard}: Dann werden wir auf der gleichen Seite stehen.

\textbf{Jerevan}: Setzt euch und füll die Becher, Lutz!
Gibt es gutes Plündergut zu holen bei den Tiburern?

\textbf{Bastard}: Es geht. Die neun Jahre haben das Land ausgezehrt und jeder nimmt das, was er kriegen kann. Zwei so hübsche unbeschädigte Bäuerinnen wie ihr sie habt, findet man nur noch selten.

\textbf{Jerevan}: Das einzige, was ich  auf dem Weg hier her gefunden habe. Schon ein wenig armselig.

\textbf{Magdalena}: Würdet ihr mich auf das Schlachtfeld begleiten?

\textbf{Jerevan}: Wiso auf das Schlachtfeld?

\textbf{Magdalena}: Ich muss meine Wäsche aufhängen.

\textbf{Jereva}: An deine Wäsche gehen wir dir noch früh genug.

\textbf{Bastard}: An diese Wäsche würde ich dir auch gerne mal gehn.

\textbf{Jerevan}: Finger, weg! Das ist meine Wäsche.

\textbf{Bastard}: Aber ein Ritterbruder wird doch wohl seine Beute teilen.

\textbf{Jerevan}: Na mit wie vielen Ritter hast du nach dem letzten Kampf dein Plündergut geteilt.

\textbf{Bastard}: Na mit mir. Ha ha ha

\textbf{Magdalena}: Aber ich muss dringend auf das Schlachtfeld, die Saat ausbringen.

\textbf{Bastard}: Etwas Samen kann ich sicherlich da beisteuern Holdeste.

\textbf{Jerevan}: Meine Furche mein Samen. Such dir selber was zum beackern.

\textbf{Magdalena}: Aber ich muss noch meine Kuh finden und melken.

\textbf{Bastard}: Mit Eutern und Zizen kenne ich mich prächtig aus. Da hat sich noch keine Kuh beschwert.

\begin{quote}
  \textit{- Jerevan seuftst entnervt, lehr den Rest vom Becher in einem Zug und packt ihn weg -}
 \end{quote}


\textbf{Jerevan}:
zu Magdalena - Na wenns unbedingt jetzt sein muss.
zum Bastard - Meine Kuh, meine Milch. Machs dir hier ruhig bequem, ich bin bald wieder zurück.


\begin{quote}
  \textit{- Abgang Jerevan und Magdalena. Derweil sind Lutz und die Zofe anscheinend eingeschlafen und der Bastard wähnt sich unbeobachtet -}
 \end{quote}


\textbf{Bastard}: Wenn das zwei Bauersfrauen sind, dann bin ich der neue König von Grenzbrueck.
Na vielleicht finde ich in ihrem Korb etwas, dass ein wenig Licht in die Sache bringt.
Ah … sieh da ein Siegelring mit dem Wappen von … Hohenstaaden. Die Frau des Großherzogs alleine auf dem Schlachtfeld. Nur beschützt von einem zerlumpten, fahrenden Ritter und seinem Knappen. Ha, wenn sich da kein Vorteil draus schlagen lässt.

\begin{quote}
  \textit{- Abgang Bastard Abgang Bühne links mit dem Ring -}
 \end{quote}


\textbf{Zofe}: Nein, er hat es bemerkt, was sollen wir nur tun?

\textbf{Lutz}: Wir werden ihm erst einmal folgen und sehen, wo er hingeht. Dann werden wir es meinem Herren sagen.

\begin{quote}
  \textit{- Abgang Lutz und Zofe. Abgang Bühne links hinter dem Bastard her -}
 \end{quote}

- hier evtl. Pause einfügen. Ist allerdings meist nicht nötig, da das Stück kurz genug ist um auch ohne Pause gespielt werden zu können. -

\begin{quote}
  \textit{- Auftritt Bühne rechts Tessa von Ravur -}
 \end{quote}


\textbf{Tessa von Ravur}: Dieser Krieg zehrt das Land auf. Kein Dorf, das nicht geplündert, keine Stadt unberührt und das alles ob unserer menschlichen Begierden. Wieviele treue und tapfre Diener muss ich noch in den Tod schicken, bis ihr Blut und das unsre, Ehre und Pflicht genüge getan hat. Ihr Ewgen, wieviel Leid muss unser Volk noch erdulden bevor es für unsre Taten gebüßt hat?

\begin{quote}
  \textit{- Auftritt Bühne links Bastard Lutz und Zofe. Lutz und Zofe schleichen hinter dem Bastard her und verstecken sich vor den beiden -}
 \end{quote}

\textbf{Bastard}: Ah meine geliebte Mutter, ich habe habe euch gesucht, an diesem schönen Abend.

\textbf{Tessa von Ravur}: Was willst du hier, Bastard?

\textbf{Bastard}: Das bin ich, meine Dame. Aber habe ich nicht im Namen meines Vaters, eures verstorbenen Ehemannes, ein wärmeres Willkommen verdient?

\textbf{Tessa von Ravur}: Mein Mann war zehn mal, der Mann der ihr seid, Bastard von Malmaise.

\textbf{Bastard}: Und nun ist er zehnmal so tot und ich bin noch hier, um euer blutiges Tagewerk zu tun.

\textbf{Tessa von Ravur}: Für solch Taten seid ihr gerade gut genug. Sagt was ihr wollt oder schert euch von dannen.

\textbf{Bastard}: Ich will das was mir zusteht, mein Erbe, das Erbe meines Vaters.

\textbf{Tessa von Ravur}: Euch steht nichts zu Bastard! Auch wenn mein Mann und alle meine Kinder tot in diesem Krieg gefallen sind, so werdet ihr niemals erben, was nicht euer ist. Was in Acrulons Namen lässt euch denken, dass ich meine Meinung ändern werde.

\textbf{Bastard}: Weil ich euch bringe, was sich euer Herz am meisten wünscht, Friede!

\textbf{Tessa von Ravur}: Friede! Aus eurer blutgen Hand. Ihr seid noch mehr von Sinnen als sonst, wenn ihr glaub das erreichen zu können, was neun Jahre Krieg nicht geschafft haben.

\textbf{Bastard}: Ich sehe heute viel klarer als jemals zuvor. Versprecht mir mein Erbe und ich werde euch den Frieden bringen.

\textbf{Tessa von Ravur}: Sag wie du das bewerkstelligen willst oder tritt aus meinen Augen.

\textbf{Bastard}: Gebt mir erst euer Versprechen.

\textbf{Tessa von Ravur}: Für den Frieden. Ja … für den Frieden würde ich sogar euch als Erben anerkennen, aber wenn ihr versucht mich zum Narren zu halten, wirds dir schlecht ergehen, auch das verspreche ich dir Bastard.

\textbf{Bastard}: Dazu werdet ihr keinen Grund haben … Mutter, - denn ich habe mit eigenen Augen gesehen, wie Margaret von Hohenstaaden alleine nur von einem einzigen fahrenden Ritter bewacht auf dem Schlachtfeld unterwegs ist. Wenn wir sie gefangen nehmen können wir ihren Mann dazu bringen aufzugeben, den Tiburern ein Stück Land abtreten und der Krieg ist vorbei.

\textbf{Tessa von Ravur}: Ihr lügt. Sie würde nie alleine hier in der Schlacht sein.

\textbf{Bastard}: Ihr seid schnell mit eurem Urteil bei der Hand … Herrin. Denn wenn sie nicht hier wäre, woher habe ich dann das hier? (Zeigt ihr den Siegelring)

\begin{quote}
  \textit{- Tessa von Ravur Sprich nun zum Publikum und der Bastard steht hinter ihrer Schulter und flüstert ihr ins Ohr. Es soll klar für das Publikum erkennbar sein, dass dies eigentlich ein innerer Dialog von Tessa von Ravur ist -
}
 \end{quote}

\textbf{Tessa von Ravur}: Ein Siegelring … mit dem Siegel von Hohenstaaden? Dann ist es wirklich war. Aber das was ihr vorschlagt ist Hochverrat. Hand an die Frau des Großherzogs, meines Lehnsherren zu legen … und sei der Grund auch noch so nobel. Wie kann aus schlechter und schändlicher Tat Frieden folgen? Meine Ehre, meine Pflicht verbieten es … ich kann die Ehre meines Blutes nicht so beschmutzen.

\textbf{Bastard}: Ihr meint die Ehre des Blutes eures Gemahls, eurer Kinder und eures Volkes, das der Großherzog Eimerweise verschüttet hat, für Gründe die schon seit Jahren null und nichtig sind. Er, der immer noch weiterkämpft, obwohl alles für das wir einst am Anfang das Schwert erhoben haben, sowieso nicht mehr zu retten ist. Er, der diesen sinnlosen Krieg immer noch weiterführt, weil er zu stolz ist zuzugeben, dass man ihn nicht mehr gewinnen kann, selbst wenn wir die Tiburer besiegen und er, der dafür jeden von uns opfern würde, dem seid ihr noch verpflichtet?

\textbf{Tessa von Ravur}: Ja dem bin ich verpflichtet mit meiner Ehre bis zum letzten Atemzug. Meine Hand gegen sein Blut erheben, wo ich doch bei den Ewgen geschworen habe ihm treu zu sein? Undenkbar!

\textbf{Bastard}: Ihr seid eurem Volk verpflichtet, eurer Familie, eurem Blute. Denen, die Gefallen sind und denen, die noch leben, aber sicher auch sterben werden, wenn dieser Wahnsinn weiter geht. Gebt mir 50 Ritter und noch vor dem Morgengrauen wird Margaret von Hohenstaaden in euren Händen und der Frieden gewonnen sein.

\textbf{Tessa von Ravur}: Neun Jahre des Krieges. Jahre in denen unsere Felder über und über getränkt werden mit unserem Blut. Wenn die Morgenröte mich zwingt wieder auf ein Schlachtfeld zu treten und mehr gute Männer, Söhne, zu opfern. Die Schreie der Trauernden und das Stöhnen der Sterbenden sind zu unseren Schlafliedern geworden. Um des Friedens willen… werde ich alles Verraten um dieses Leid zu beenden. Ja, Bastard eure Wünsche sollen euch erfüllt werden. Ihr werdet 50 Mann bekommen und bringt mir Margaret von Hohenstaaden!

\begin{quote}
  \textit{- Die Zofe zerbricht einen Stock. Tessa von Ravur und der Bastard werden aufmerksam -}
 \end{quote}


\textbf{Zofe}: Nein, sie wollen meine Herrin gefangen nehmen.

\textbf{Lutz}: Pscchhht, leise.

\textbf{Bastard}: Was?

\textbf{Tessa von Ravur}: Wer ist da? Wer lauert da im dunklen Gestrüpp.

\textbf{Bastard}: Das ist dieser verdammte Knappe mit der Bauersmagt.

\textbf{Tessa von Ravur}: Sie dürfen davon nichts berichten. Töte sie! Ich komme mit den Rittern nach.

\textbf{Bastard}: Das mache ich, bleibt stehen ihr Würmer!

\begin{quote}
  \textit{- Es beginnt eine Verfolgungsjagt. Der Bastard verfolgt die Zofe und Lutz. Diese Verfolgungsjagt soll lustig gestaltet werden, damit die Zuschauer zwischen den beiden anspruchsvollen Szenen auch etwas zu Lachen bekommen.
Diese Szene muss stark an die örtlichen Gegebenheiten angepasst werden und endet wie folgt: -}
 \end{quote}


\textbf{Lutz}: Wir können vor ihm nicht ewig weglaufen, wir müssen eine Falle stellen.

\textbf{Zofe}: Aber wie denn nur?

\textbf{Lutz}: Was hast du dabei?

\textbf{Zofe}: Nur was in den Korb ist, diesen Stickrahmen.

\textbf{Lutz}: Ich hab eine Seil und das Tuch zum polieren der Rüstung. Warte, damit können wir ihm eine Falle stellen.

\begin{quote}
  \textit{Dann verstecken sie sich hinter zwei “Bäumen” und spannen das Seil. Der Bastard tapt herein und fällt. Lutz wirft ihm das Poliertuch über den Kopf und die Zofe stülpt ihm den Stickrahmen drüber, damit es fest bleibt. Während der Bastard blind herumtorkelt wickeln die beiden ihn in das Seil ein, fesseln ihn somit kurzzeitig und ergreifen die Flucht.
Der Bastard flucht laut, tölpelt etwas herum, entledigt sich schließlich der Gegenstände und geht dann fluchend von der Bühne ab.
}
 \end{quote}

\textbf{Bastard}: Ich kriege euch ihr Maden.


\begin{quote}
  \textit{- Abgang Bastard rechts -
}
 \end{quote}

\begin{quote}
  \textit{- Auftritt Tiburer mit Leiche (Puppe) im Arm von links. Er legt die Leiche zentral auf der Bühne direkt vor der Rückwand ab, torkelt etwas und geht auf halbem Weg zwischen Puppe und Aufgang rechts im Frontbereich zum Publikum zu Boden. -
}
 \end{quote}

\begin{quote}
  \textit{- Aufgang Jerevan und Margaret von Rechts Arm in Arm -
}
 \end{quote}

\textbf{Jerevan}: Guck mal da, hinter diesem dunklen dichten Gebüsch … da ist sicher deine Kuh.

\textbf{Margaret}: Seht da, ein verwundeter Ritter.

\textbf{Jerevan}: Ach, nicht noch einer. Jetzt sag nicht, um den willst du dich auch noch kümmern. Die Euter deiner Kuh müssen doch schon fast am bersten sein.

\textbf{Margaret}: Ich muss mich doch um ihn kümmern.

\textbf{Jerevan}: Nein, das musst du nicht. Das ist eh ein Tiburer, wenn ich ihm morgen auf dem Feld begegne, dann werde ich ihn sowieso töten. Dass kann er auch jetzt haben...

\begin{quote}
  \textit{- Jerevan zieht sein Schwert -
}
 \end{quote}


\textbf{Jerevan}: Schneller und Schmerzloser.

\textbf{Margaret}: Nein!

\textbf{Jerevan}: Ach, was für eine dämliche Idee das gewesen ist euch beide mitzunehmen. Wir laufen jetzt schon die ganze Nacht hier herum und du hast nichts besseres zu tun, als dich um diese elenden Ritter zu kümmern, die nicht gut genug waren um auf sich aufzupassen und sowieso bald vor dem Fährmann stehen werden. Ach verdammte Scheiße …

\textbf{Tiburer}: Vielen Dank für eure Hilfe meine Dame. Ich verdanke euch mein Leben, ich stehe tief in eurer Schuld, bei meiner Ehre, wenn ich etwas …

\textbf{Jerevan}: Verschwinde du Idiot und sei froh, dass du noch lebst. Ich muss noch eine Kuh finden.

\begin{quote}
  \textit{- Jerevan tritt den Tiburer. Abgang Tiburer Bühne rechts -
}
 \end{quote}

\begin{quote}
  \textit{- Auftritt Bühne links Lutz und Zofe -
}
 \end{quote}

\textbf{Jerevan}: Halt wer … Ach du bist es. Unangekündigt in mein Schwert zu laufen ist ein guter Weg …

\textbf{Lutz}: Herrin, Herrin, sie kommen um euch zu holen, sie haben es herausgefunden und wollen euch gefangen nehmen um …

\textbf{Jerevan}: Das heißt Herr du Idiot und wer kommt um mich zu holen und was redest du überhaupt?

\textbf{Lutz}: Herrin, Großherzogin von Hohenstaaden, die haben herausgefunden, wer ihr seid und wollen euch entführen, um euren Mann den Großherzog von Hohenstaaden zu erpressen.

\textbf{Jerevan}: Ach, verdammte Scheiße.

\begin{quote}
  \textit{- Auftritt Bastard Bühne links, die Zofe läuft zu ihrer Herrin und betüddelt sie -
}
 \end{quote}

\textbf{Bastard}: Ja genau.

\textbf{Jerevan}: Bleib genau da stehen, wenn du weiterleben willst.

\textbf{Bastard}: Was willst du denn? Sag, was willst du für die Frau, die dich die ganze Zeit zum Narren gehalten hat?

\textbf{Jerevan}: Bäuerin oder nicht, ich hab ihr mein Wort gegeben.

\begin{quote}
  \textit{- Der Bastard wirft ihm einen Geldbeutel zu, der an Jerevans Brust abprallt und zu Boden fällt -
}
 \end{quote}

\textbf{Bastard}: Ich bezahle für dein Wort, mehr als ein fahrender Ritter jemals hoffen kann zu besitzen.

\textbf{Jerevan}: Mein Wort kann man mit keinem Gold der Welt kaufen …

\textbf{Bastard}: … aber mit Stahl erschlagen.

\begin{quote}
  \textit{- Bastard und Jerevan kämpfen. Jerevan besiegt den Bastard. Hier ist eventuell Platz für Rüstungs oder Schmiedeangebote -
}
 \end{quote}

\textbf{Jerevan}: Pah, Idiot. Und nun zu uns.

\begin{quote}
  \textit{- Auftritt Bühne links Tessa von Ravur mit Rittern -
}
 \end{quote}

\textbf{Tessa von Ravur}: Was auch immer er gewesen war … Ritter. Er war von meinem Blute und du hast ihn erschlagen. Dafür wirst du sterben.

\textbf{Jerevan}: Jerevan von Eibenhain gegen 50 Ritter? Na das ist doch mal ein Kampf von dem es sich lohnen wird zu berichten.

\begin{quote}
  \textit{- Margaret entdeckt die Puppe -
}
 \end{quote}

\textbf{Margaret}: Neiiiin, mein Mann … tot danieder!

\textbf{Tessa von Ravur}: Verdammt, damit ist der Frieden dahin und der Morgen wird das Schlachtfeld erneut rot färben.
Vor meinem toten Lehnsherren werd ich dich nicht erschlagen … Ritter, aber wenn die Sonne aufgeht, dann bist du tot. Ihr da! Nehmt ihn mit.

\textbf{Tessa von Ravur}: Verdammt, damit ist unsere Hoffnung dahin und die anderen Feldherren werde sich nicht zu einem Frieden umstimmen lassen. So wird auch dieser Morgen das Schlachtfeld erneut rot färben. Vor meinem toten Lehnsherren werd ich dich nicht erschlagen, doch wisse so denn die Schlacht beginnt, so wird es keine Gnade geben. Dann bist du tot!

\begin{quote}
  \textit{- wirft den Siegelring auf Jerevan, der von seiner Brust abprallt -
}
 \end{quote}

Bis dahin Ritter … Jerevan von Eibenhain. - Ihr da! Nehmt ihn mit.

\begin{quote}
  \textit{- Abgang links Tessa von Ravur mit Rittern und Leiche vom Bastrad -
}
 \end{quote}


\textbf{Jerevan}: Ach scheiße.

\textbf{Lutz}: Herr? Was sollen wir nun machen?

\textbf{Jerevan}: Hm, wenn wir schnell handeln, dann können wir uns vielleicht durch die Tiburer Linien schleichen und irgendwie wegkommen.

\textbf{Lutz}: Herr, aber ihr sagtet doch, dass die TIburer uns niemals aufnehmen werden.

\textbf{Jerevan}: Ja, das werden sie auch nicht, wenn sie uns erwischen, dann sind wir tot. Sie sind bessere Reiter und und kennen das Land. Unsere Überlebenschancen gibt es nicht wirklich.

\textbf{Lutz}: Wenn wir nicht überleben können, vielleicht sollten wir dann hier bleiben.

\textbf{Jerevan}: Hast du sie nicht gehört? Hier sind wir garantiert tot, in Tibur haben wir vielleicht noch eine Möglichkeit.

\textbf{Lutz}: Aber ihr habt ihr doch euer Wort gegeben.

\textbf{Jerevan}: Ich wollte die Kuh einer Bäuerin suchen und sie ist nur die ganze Zeit übers Schlachtfeld gelaufen und hat Verwundete geheilt, von beiden Seiten … dumme Kuh … dankbar erweisen … wenn ich nicht lache.

\textbf{Lutz}: Ihr habt gelobt ihr bei ihrer Aufgabe zu helfen und sie wollte Frieden für das Land.

\textbf{Jerevan}: Und wie soll ich das erreichen?

\textbf{Lutz}: Die Ritter denen sie geholfen hat, die waren doch bestimmt dankbar, oder?

\textbf{Jerevan}: Ja klar, die noblen aufgepufften Edelmänner haben bei ihrer Väter Väter Ehre geschworen Ihre Schuld … blablabla

\textbf{Lutz}: Wie wärs, wenn wir zu diesen Ritter …

\textbf{Jerevan}: Ja, vielleicht könnte uns einer von denen durch die feindlichen Linien …

\textbf{Lutz}: Herr …

\textbf{Jerevan}: Halts Maul, ich muss nachdenken. … Du Lutz ich habe eine Idee. Als wir übers Schlachtfeld sind hat die Herzogin Ritter von beiden Seiten geheilt, die ihr jetzt ihr Leben schulden … ich hab eine Idee … komm mit.

\textbf{Lutz}: Ja, wunderbar Herr.

\begin{quote}
  \textit{- Abgang Bühne links Jerevan und Lutz -
-Margareth und die Zofe bleiben auf der Bühne, Margareth hinter dem Umhang der Zofe, die mit dem Rücken zum Publikum steht. Sie zieht eine weiße Perücke auf und bleibt hinter dem Umhang versteckt.-
- Auftritt Bühne rechts das Heer von Tibur, Auftritt Bühne links das Heer von Hohenstaaden -
}
 \end{quote}

\textbf{Teoderederederich}: Und so treffen die Heere erneut aufeinander. Reite Tibur reite. Eomund du meine beste Lanze, du wirst den ersten Stoß führen.

\textbf{Eomund}: Mein Fürst Teoderederederich ich kann nicht. Bei meiner Väter Väter Väter Väter Ehre Karren habe ich geschwohren im heutigen Kampf weder Lanze noch Schwert zu erheben gegen unseren Feind.

\textbf{Teoderederederich}: Eodmund mein treuer Reiter. Du wars immer treu und so werde ich nichts verlangen, was wider deine Ehre ist.

\textbf{Tessa von Ravur}: Dann wirst du Gisber ehrenvollster Ritter Hohenstaadens die Schlacht beginnen. Spanne die Armbrust und schleiche dich um ihre Flanke.

\textbf{Gisber}: Herrin auch ich habe geschworen in diesem Kampfe nicht die Armbrust gegen den Feind zu erheben.

\textbf{Teoderederederich}: Dann wirst du Helmhammerhaar den Angriff führen.

\textbf{Helmhammerhaar}: Mein Fürst auch ich habe geschworen.

\textbf{Tessa von Ravur}: Dann du Juliette.

\textbf{Juliette}: Herrin auch ich habe geschworen die Armbrust ruhen zu lassen.

\textbf{Teoderederederich}: Dann du Erzsteintrutzheim.

\textbf{Erzsteintrutzheim}: Auch ich, mein Fürst.

\textbf{Tessa von Ravur}: Du auch Fleur?

\textbf{Fleur}: Auch ich.

\textbf{Tessa von Ravur und Teoderederederich}: Wem habt ihr das geschworen?

\textbf{Alle}: Ihr haben wir das geschworen.

\begin{quote}
  \textit{Margaret hat sich derweil unter dem Umhang ihrer Zofe umgezogen mit der Perücke mit den weißen Haaren und kommt nun als göttlich gesanndte heilige Margaret zentral zwischen den beiden Parteeien hervor.}
 \end{quote}


\textbf{Tessa von Ravur und Teoderederederich}: Ein Wunder.

\textbf{Margaret}: Das Land hat genug geblutet. Ob gerechter Streit oder noble Sache, für die eine Seite, oder die Andere sei gleich. Barmherzigkeit und Mildtätigkeit ist Tugend eines jeden Ritters und 9 Jahre des Blutvergießens sind genug. Steckt die Schwerter ein, betrauert die Toten und Ehrt ihr Opfer. Geht nach Hause zurück zu euren Familien, baut das Land wieder auf und seid Barmherzig zu euren einstigen Feinden.

\textbf{Alle}: Es sei Frieden.

\begin{quote}
  \textit{- Abgang Alle in einer Kolonne durch die Mitte -
}
 \end{quote}


\chapter{Die legende von Blut und Feuer}

\section{Prolog auf der Erde}

\textbf{Eria:} He, wer da? Im Name des Ewigen und seiner Töchter, gib dich zu erkennen!

\textbf{Merolie:} Halt, nicht schießen, ich bin es.

\textbf{Eria:} Merolie?

\textbf{Merolie:} Ja. Keine Angst, die Nekaner wagen keinen Angriff in der Nacht. Erst morgen früh
wird es soweit sein.

\textbf{Eria:} Bei deinem Wort, wir haben sie doch längst in der Stadt.

\textbf{Merolie:} Nein die Männer aus Caldros verehren zwar den Flammenden und sind von ihrem
Blut, doch sind es Flüchtlinge, die nach Hilfe suchen.

\textbf{Eria:} Und dann sag mir nochmal, warum in Furathas Namen wir sie ihnen gewähren sollten.

\textbf{Merolie:} Na, weil sie gut bezahlen. Außerdem ist ihre Prinzessin Ilea verheiratet mit Adran
Himmelsturm. Sie werden Schiffe kriegen und sie werden von dem Navigator in das
Land geführt, aus dem die Himmelsstürmer vor Jahrzehnten zu uns kamen.

\textbf{Eria:} Wenn es nach mir geht, können sie nicht schnell genug weg sein. Sollen sie doch
fliehen, die Feiglinge.

\textbf{Merolie:} Ein paar von unsrem Volk fliehen auch. Sie haben Angst vor Tiotep, dem
Unbesiegten. Er ist der Sohn des Desrutep und führt die Truppen in die Schlacht.
Niemand vermag dem schönen Prinz des Krieges zu wiederstehen.

\textbf{Eria:} Soll er nur kommen, gegen so einen Gegner wie uns hat er noch nicht gestanden. Ich
kann ihn sehen dort unten im Feldherrenzelt, wie er schon jetzt vor Angst zittert.

\section{Prolog der Götter 1:}

\textbf{Tiotep:} Ach, ich sage dir, morgen, das wird ein Sieg werden, wie er der roten Schwadron
würdig ist. Die Händler hatten kein Herz und keine Seele, keine Flammen und keine
Ehre. Sie waren es nicht würdig, gegen den Prinzen des Krieges und das Nekanische
Heer zu kämpfen. Unser Sieg war ohne Belang. Aber diese Menschen in Drana, sie
tragen ihre Herzen hoch. Sie werden würdige Gegner sein, wenn wir sie morgen
besiegen.

\textbf{Diener:} Herr, Alatep, der Sohn des Justotep, wünscht euch zu sprechen.

\textbf{Tiotep:} Komm herein, Cousin und Bruder in der Schlacht. Komm und feier mit mir unseren
morgigen Sieg.

\textbf{Alatep:} Gerne, Tiotep, Prinz des Krieges, aber sag mir, feiert man einen Sieg nicht erst, wenn
man ihn errungen hat?

\textbf{Tiotep:} Ach dummes Gewäsch, was für einen Sinn hat es, in eine Schlacht zu ziehen, wenn
man nicht siegen wird? Du verbringst zuviel Zeit mit den Generälen meines Vaters.

\textbf{Alatep:} Und du zuwenig. Wieder einmal wurdest du an dem Feldherrentisch Pyrdracors
vermisst. Dein Vater ist es schon fast gewöhnt, seine Kriege ohne dich zu planen.

\textbf{Tiotep:} Und ich bin es gewöhnt, sie ohne seine Pläne zu gewinnen. Aber was macht Alatep,
der Philosoph und Politiker eigentlich hier? Ich dachte, dir würde es reichen, die
Staaten zu lenken, die ich erschaffe.

\textbf{Alatep:} Ich lenke, du erschaffst, darüber wollte ich mit dir reden. Aber alleine

\begin{quote}
Blick auf den Diener auf einen Wink von Tio verlässt dieser den Raum
\end{quote} 

\textbf{Tiotep:} Alleine, der gerechte Alatep hat Geheimnisse? Und was kommt als Nächstes? Willst
du etwa deine Steuern nicht bezahlen?

\textbf{Alatep:} Ich bin rastlos und zu dir gekommen, weil dunkle Gedanken mich plagen. Dieser
Kireg ist nicht gerecht.

\textbf{Tiotep:} Ähhh lacht

\textbf{Alatep:} Wie sollen wir den Menschen Gerechtigkeit bringen, wenn wir selber keine
Gerechtigkeit üben? Wie können wir mit mächtiger Hand einen schwächeren Feind
einfach niederringen und es dann Gesetz und rechtens nennen?

\textbf{Tiotep:} Na wem sagst du das. Das war doch viel zu einfach bis jetzt. Gib mir mal eine richtige
Herausforderung.

\textbf{Alatep:} Ich glaube, das könnte ich. Erinnerst du dich, Bruder, an jene aus Caldros? Ein
kleines Händlervolk aus Alinos. Tief in ihren Herzen glauben sie an Pyrdracor, doch
haben unsere Väter entschieden, sie mit Feuer zu schlagen wie alle anderen auch.
Sie sahen keinen Ausweg mehr als zu fliehen. Ihre Schiffe liegen nun im Hafen von
Drana. Und ihre Edlen beraten in dieser Stunde mit den Hohen von Drana darüber, ob
sie bleiben sollen und sich dem sicheren Untergang stellen oder übers Meer fliehen.
Ein Navigator aus Drana, der die Prinzessin aus Caldros geehelicht hat, hat angeboten,
sie über das Meer zu führen in eine neue Heimat, wo sie frei leben können. Wie kann
es sein, dass unsere Taten die gläubigen Kinder dazu treiben, sich mit den Dienern des
Nachtblauen einzulassen und mit ihnen Bande zu knüpfen? Dass sie sich mit ihnen
verbünden, um ihr Leben zu retten? Zu retten vor uns, die wir doch kommen um ihnen
den Glauben und das Leben zu bringen?

\textbf{Tiotep:} Du hast recht! Auch wenn sie uns verraten haben, so werden sie gute Gegner sein und
einen würdigen Kampf abgeben. Und was soll das Gerede vom Überleben? Welchen
Wert hat ein Leben, das nicht gelebt wird? Lebe in deinem Haus, bestelle dein Land
und siehe zu, wie deine Kinder aufwachsen. Oder sterbe jung an Jahren in der Hitze
der Schlacht mit dem Feuer in den Augen und der heißen Glut im Herzen. Wäre solch
ein Leben voll Feuer, wenn auch kurz an Jahren, nicht so unendlich viel wertvoller als
ein Dutzend anderer.

\textbf{Alatep:} Ich wünschte mir, mein Herz würde so schnell schlagen wie das deine und mein Geist
würde so heiß brennen wie der deine. Brennen nur im Hier und Jetzt. Heiß jede
Sekunde leben, ohne an das Morgen zu denken. Doch ich bin geschlagen mit einem
grübelndem Geist, der einem Köhlerfeuer gleich oft tagelang schwelt und nicht zu
atmen weiß wie du es vermagst. Und wäre ich wie Tiotep und würde leben für den
Kampf, so müsste mein Geist mich doch immerzu bremsen und mich fragen: Und
morgen? Wenn diese letzte große Schlacht geschlagen ist? Wer wird mir morgen
gegenüberstehen? An wessen Schwert kann ich mich morgen messen, wo dies doch
die letzte Schlacht sein wird, die für lange Zeit geschlagen wird? Du weißt es besser
als ich. Wenn die Sonne morgen untergeht, wird Drana besiegt sein. Danach steht nur
noch Sundan und dort wird der Krieg in Gräben und hinter Mauern geführt und nicht
auf dem Schlachtfeld.

\begin{quote}
Tiotep seufzt, blickt in die Ferne und gübel
\end{quote}


Du weißt, das ich recht habe.

\textbf{Tiotep:} Und was genau willst du, was ist dein Plan?

\textbf{Alatep:} Wenn das Volk aus Caldros aufbricht, will ich sie begleiten und mit ihnen ein neues
Reich gründen. Sie vor den Worten des nachtblauen Drachen bewahren und ihnen
Vorbild und Leitstern sein. Dort in der Ferne warten Abenteuer. Nicht hier in Neka.

\textbf{Tiotep:} Das hört sich gut, doch kämpfen wir hier und Destrutep wird seinen größten Krieger
nie übers Meer schicken, wenn hier auch noch Kriege warten. So langweilig sie auch
sein mögen.

\textbf{Alatep:} Ich denke schon, lass uns zu ihm gehen und dies das Problem des Politikers und
Philosophen sein.

\textbf{Tiotep:} Mit seinem Segen den Kampf gegen Barbaren und die Einflüsterungen des
Nachblauen aufnehmen. Ja, das könnte ein Abenteuer sein, das Tiotep würdig ist.

\begin{quote}
Tiotep ab. Alatep noch alleine auf der Bühne
\end{quote}


\textbf{Alatep:} Destrutep wird zustimmen. Er weiß um das Heißblut seines Erstgeborenen und wie
wenig es ins neue Reich der Legionäre passen würde.

\section{Prolog der Götter 2:}

Die beiden Wachen gehen wieder über die Stadtmauer. Dort stehen auch die beiden
Göttinnen, doch die Wachen sehen sie nicht.

\textbf{Eria:} He, wer da? Im Name des Ewigen und seiner Töchter, gib dich zu erkennen!

\textbf{Merolie:} Halt, nicht schießen, ich bin es.

\textbf{Eria:} Merolie?

\textbf{Merolie:} Ja. Keine Angst, die Nekaner wagen keinen Angriff in der Nacht. Erst morgen früh
wird es soweit sein.

\textbf{Eria:} Bei deinem Wort, wir haben sie doch längst in der Stadt.

\textbf{Merolie:} Nein die Männer aus Caldros verehren zwar den Flammenden und sind von ihrem
Blut, doch sind es Flüchtlinge, die nach Hilfe suchen.

\textbf{Eria:} Und dann sag mir nochmal, warum in Furathas Namen wir sie ihnen gewähren sollten.

\textbf{Merolie:} Na, weil sie gut bezahlen. Außerdem ist ihre Prinzessin Ilea verheiratet mit Adran
Himmelsturm. Sie werden Schiffe kriegen und sie werden von dem Navigator in das
Land geführt, aus dem die Himmelsstürmer vor Jahrzehnten zu uns kamen.

\textbf{Eria:} Wenn es nach mir geht, können sie nicht schnell genug weg sein. Sollen sie doch
fliehen, die Feiglinge.

\textbf{Merolie:} Was soll denn daran feige sein, sie wollen doch nur überleben. Göttin Creatha würde
ihnen sicher einen guten Neuanfag gönnen.

\textbf{Creatha:} Ja, das werde ich.

\begin{quote}
Creatha vorher und nachehr im Freeze. Die Wachen bemerken sie nicht.
\end{quote}

\textbf{Eria:} Furatha wird uns zum Sieg verhelfen. Du willst einen Neuanfang und dich vor ihnen
verstecken. Wir werden auf ihren Gräbern tanzen.

\textbf{Merolie:} Ja, Gräber wird es morgen geben, aber ich bezweifle, dass jemand tanzen wird.

\begin{quote}
Wachen ins Freeze
\end{quote}


\textbf{Furatha:} Da, Siehst du, was du und deine verfluchten Caldrier mit dem Gerede von Flucht
angerichtet habt. Selbst unsere eigenen Leute vertrauen schon nicht mehr auf unseren Sieg.

\textbf{Creatha:} Ach Furatha, Göttin der Leidenschaft und Raserei, Tochter des Gottdrachen
Hydracor, Schwester. Was willst du diesen Menschen einen Neuanfang, ein Überleben
verwehren?

\textbf{Furatha:} Sie sollen hier bleiben und kämpfen für das, was ihnen wichtig ist. Adran
Himmelsturm sollte besser wissen, als unsere Seele an die Diener des Flammenden zu
verraten und mit der Prinzessin aus Caldros das Lager zu teilen. Es ist seine Feigheit,
die unsere Soldten zweifeln lässt.

\textbf{Creatha:} Es ist nicht der kühne Plan des Navigators, der Zweifel gesät hat, sondern der nahe
Untergang. Das, was ihnen wichtig ist, können sie auch woanders neu erschaffen.
Furatha kann wie ein Blitz im Augenblick existieren, aber Menschen brauchen das
Morgen, brauchen Hoffnung.

\textbf{Furatha:} Er ist dort unten, Creatha, Er wird morgen den Angriff führen.

\textbf{Creatha:} Wer, Tiotep?

\textbf{Furatha:} Ja, Tiotep der löwenhäuptige, aufgeblasene Prinz des verdammten Krieges. Ich
werde ihn zerquetschen wie eine reife Traube.

\textbf{Creatha:} Ach, Furatha. Wieso nur? Wieso musstest du ihn erwählen?

\textbf{Furatha:} Was willst du damit sagen?

\textbf{Creatha:} Götter, Halbgötter und Sterbliche liegen dir zu Füßen. Sie alle lieben dich aufrichtig
und sehnen sich danach, würdest du auch nur ein wenig für sie empfinden. Doch du
verschwendest deine Liebe an den einzigen Mann, der dich niemals lieben wird.

\textbf{Furatha:} Das tue ich nicht...

\textbf{Creatha:} Du lügst und du weißt es. Aber Tiotep wird morgen nicht kämpfen. Er und Alatep
werden mit diesen fliehenden Händlern aus Caldros gehen, die gerade in der Stadt
sind und in der Ferne einen neuen Anfang machen. Er wird mit dem Navigator Adran
Himmelsturm die Schiffe besteigen und diese Gestade verlassen. Ich werde ebenfalls
mitgehen, denn einige Wenige unseres Volkes fliehen auch. Ich werde ihnen eine neue
Heimat geben.

\textbf{Furatha:} Verflucht seist du, Tiotep. So einfach entkommst du mir nicht.

\section{1.Akt 1. Szene:}

Im Hafen in Drana. Adran Himmelsturm, Prinzessin Ilea und Hafenarbeiter. Tio, Ala, Crea
und Furata besteigen das Schiff. Es kommt zu ersten Reibereien. Adran ringt den Göttern
das Versprechen ab, sich zu benehmen und macht den großen Monolog, was sie überhaupt
vorhaben und warum.

\begin{quote}
Tiotep ist verkleidet und ein Matrose stellt ihn der Prinzessin vor
\end{quote}


\textbf{Matrose:} Prinzessin dieser wandernde Schwertmeister will uns begleiten.
Tiotep:Edle Ilea, eure Reise fürt euch ins unbekannte Abenteuer. Ich werde euer Schwertarm
sein und euer Lächeln meine Sonne. Für euch werde ich die Barbaren in den fernen
Landen bezwingen und ihre Sterne in eure Hände legen, so dass ihre fremden Götzen
vor Neid vergehen

\textbf{Ilea:} Wohl gesprochen, werter Schwertmeister und gerne nehme ich euer großzügiges
Angebot an, doch wisset, dass meine Sonne und Sterne bereits in der Hand des
Navigators liegen. Mein Gemahl wird sicherlich über eure Teilname so erfreut sein,
wie ich es bin, denn unser Vorhaben braucht tapfere Männer wie euch.

\begin{quote}
Furatha, ebenfalls verkleidet, drängt sich an den beiden vorbei
\end{quote}


\textbf{Furatha:} Männer brauchen wir, ja, doch keine selbstverliebten Knaben deren einziges
Werkzeug ihre lüsterne Zunge ist, die sie endlos in nichtssagenden Worthülsen
schnalzen.

\begin{quote}
Adran Himmelssturm kommt mit Alatep(verkleidet als Justotep Priester) und Creatha
(ebenfalls verkleidet) hinzu
\end{quote}


\textbf{Adran:} (spricht mit Alatep und Creatha) So sei es wie wir besprachen. Seid bereit, denn wir
brechen bald auf.
(zu Ilea) Unsere neue Heimat wird uns alle zusammenführen. Unser Bund soll allen ein
Vorbild sein.
(dann stellt er sich mit Ilea zusammen vor die Versammenlten und hält in Richtung des Kais
(also des Publikums seine Rede)
Hört mich an Volk von Drana und Familien aus Caldros.

blabla Schwulst

Obwohl meine Sippe vor Generationen von weit her gekommen ist, so gehört mein Herz doch
dem nachtblauen Drachen der Wellen. Meine Frau

alle gehen an Bord, nur Tiotep und Adran bleiben zurrück

\textbf{Tiotep:} Navigator, ich bin ein reisender Schwertmeister, der sich gerne eurer Reise
anschließen würde. Ich hatte gerade mit eurer Gemahlin gesprochen und sie gab mir
ihren Segen.

\textbf{Adran:} Ich weiß wer ihr seid Gott Tiotep, löwenhäuptiger Prinz des Krieges und es ist mir
eine außerordentliche Ehre, dass ihr uns begleitet. Aber seid euch eines bewusst: Es
gibt Regeln, denen auch ihr euch unterwerfen müsst, wenn ihr uns begleiten wollt.

\textbf{Tiotep:} Du wagst es, Sterblicher, mir zu befehlen?

\textbf{Adran:} Nein, Gott. Dies würde ich mir niemals anmaßen. Es ist ein Codex, dem ihr euch
unterwerfen müßt und genauso wie eure Ehre ein Ideal und höchstes Gut. Der Pakt
der Wellen wird den Frieden von nun bis in alle Ewigkeit zwischen unseren Völkern
und unseren Göttern bewahren. Jeder der an Bord geht (er blickt die Planke hoch und
vorallem Alatep, Creatha und Furatha hinterher) stimmt dem Pakt der Wellen zu.

\textbf{Tiotep:} So soll es sein und nur um euch zu zeigen, dass die Söhne Pyrdracors ihre Eide halten
und sich nicht herauswinden wie die wankelmütigen Töchter deiner Wasserschlange.

\textbf{Adran:} Dann sei es so Gott, bei eurer Ehre. Dein Volk wartet auf dich und die Zukunft in die
wir es führen werden.

\section{1.Akt 2.Szene:}

Auf dem Schiff. Der Streit zwischen Furatha und Tio schaukelt sich immer weiter hoch,
genauso wie der Streit zwischen den Dranaern und den Caldriern.

\textbf{Dranaer:} Was soll das denn sein. Einen Palstek habe ich gesagt.

\textbf{Caldrier:} Ach verdammt, dann machs doch alleine. Dämliche Schiffe.

\textbf{Dranaer:} Diese dämlichen Schiffe sind alles, was uns vor deinen mordlüsternen Brüdern
gerettet hat.

\textbf{Caldrier:} Das sind Soldaten, keine Mörder, so wie eure Meuchler und Assasinen. Ohne euch
wäre es gar nicht soweit gekommen. Wenn ihr den Ewigen nicht erzürnt hättet, hätten
wir alle ruhig in Alinos weiterleben können.

\textbf{Dranaer:} Einer musste doch aufstehen und die Wahrheit sagen. Ihr hattet ja nicht den Mut
dazu.

\textbf{Caldrier:} Wir sind ja nicht dumm und müssen uns immer gegen die Mächtigeren auflehnen.

\textbf{Dranaer:} Feige seit ihr, nichts weiter.

\textbf{Caldrier:} Das von einem dieser Diebe und Meuchler. Na warte ...

\begin{quote}
will ihm eine verpassen
\end{quote}

\textbf{Adran:} Aufhören sofort! Diee Reise können wir nur zusammen bestehen. Zusammen
bestehen oder alleine untergehen. Merkt euch das.

\textbf{Dranaer:} Aye Himmelsturm.

\textbf{Caldrier:} Ja Herr.

\section{1.Akt 3.Szene:}
Creatha, Alatep und Adran versuchen, das zu klären und die Situation noch zu retten.

\textbf{Creatha:} Wir müssen sie beschwichtigen, das geht nicht mehr lange gut.

\textbf{Alatep:} Du hast recht, aber du bist ja auch die einzige, mit der man reden kann. Deine
Schwester ist genauso schlimm wie mein Bruder und die Menschen tun es ihnen nach.
Die Leidenschaft, die die beiden versprühen, zieht die Menschen in ihren Bann und
lässt sie nicht mehr los.

\textbf{Creahta:} Was schlägst du vor, du bist der Richter. Mediatha, meine Schwester, wüsste was zu
tun ist, aber ich bin nicht sie.

\textbf{Alatep:} Wir werden mit ihnen reden. Wir werden Vernunft walten lassen und wir werden
ihnen nicht sagen, dass Drana gefallen ist.

\textbf{Creatha:} Ich fürchte, dazu ist es zu spät.

\begin{quote}
Auf der anderen Seite der Bühne betritt Tiotep mit zwei Caldriern und Furatha mit zwei
Dranaern die Bühne.
\end{quote}

\textbf{Tiotep:} Ich sage dir, ich bin froh, nicht dagewesen zu sein, als Drana fiel. Das war kein
Kampf, sondern ein Gemetzel. Wie Opferlämmer haben sie sich abschlachten lassen.

\textbf{Cadrier:} Und die feigsten von den Schafen haben wir hier.

\textbf{Dranaer:} Du erbärmlicher Lügner. Vielleicht hätten wir dich gegen deine Brüder auf den
Zinnen stehen lassen sollen.

\textbf{Furatha:} Dann hätten sie vielleicht aus Wut auf ein so hässliches Gesicht mit Schwertern und
Speeren angegriffen, anstatt die Stadt aus der Ferne wie Feiglinge mit Katapulten zu
beschießen.

\textbf{Tiotep:} Der feigste Diener des Flammenden hat immer noch mehr Ehre im Leib als eure
zerlotterte Bande von Strauchdieben.

\textbf{Creatha:} Haltet ein. Stop. Wir sind hier, um einen Neuanfang zu machen und nicht, um uns
von unseren alten Rivalitäten auffressen zu lassen.

\textbf{Alatep:} Wir haben das einzig Richtige getan, als wir die Alten Lande verließen; und nun
werden wir nach Recht und Gesetz leben, dessen Gerechtigkeit uns daheim versagt
blieb. Der Pakt der Wellen wird unser Gesetz sein, bis wir ein neues Land für uns
gefunden haben.

\section{1.Akt 4.Szene:}

Es kommt zum offenen Kampf zwischen Tio und Furatha. Crea und Ala beschwören den
Wirbelsturm, der die Schiffe und damit auch die Völker trennt. Tio besiegt Furatha, Adran
stellt sich dazwischen, Tio schenkt Furatha sein Herz.

\begin{quote}
Creatha, Alatep, Furatha und Adran sitzen zusammen und beratschlagen. Tiotep sitzt mit der
Prinzessin Ilea etwas abseits und flirtet mit ihr herum.
\end{quote}


\textbf{Alatep:} Es wird immer schwerer, die beiden Völker unter Kontrolle zu halten. Jeder
Zwischenfall macht es nur noch schlimmer.

\textbf{Furatha:} Na, manchen scheint es ja regelrecht daran gelegen zu sein.
Seitenblick zu Tio

\textbf{Creatha:} Ja, dir zum Beispiel, Schwester.

\textbf{Alatep:} Wir müssen die Menschen wieder zur Räson bringen, auf dass sie auf die Gesetze
hören.

\textbf{Creatha:} Das wird nicht leicht, die Emotionen kochen hoch, nachdem Kunde zu uns drang,
dass Drana gefallen ist.

\textbf{Furatha:} Na wir wissen ja, wessen Fehler das ist. wieder Seitenblick zu Tiotep

\textbf{Adran:} Wie sollen unsere Völker denn zusammen ein neue Zukunft bauen, wenn ihre Götter
nicht einmal ihren Zwist für eine kleine Weile beiseite legen können? Ihr seit die
Herzen eurer Völker! Eurem Vorbild folgen sie.

\textbf{Furatha:} Ja! Und dieser da (zeigt auf Tiotep) sorgt dafür, das sie uns in den Abgrund folgen

\begin{quote}
Tio springt zornentbrannt auf Furatha zu. Die hat das nur erwartet und steht mit gezogener
Waffe bereit.
\end{quote}


\textbf{Tio:} Jetzt ist es genug, elende Schlange. Du bist es, die ihren Verstand mit Anarchie vergiftet
und sie kopflos in den Untergang rennen läßt. Was willst du von mir? Was?

\textbf{Creatha:} Dein Herz will sie, nicht mehr und nicht weniger.


\begin{quote}
Ein Kampf entbrennt zwischen den beiden

Während sich der Kampf zwischen den beiden in den Hintergrund verlagert und Adran Ilea in
Sicherheit bringt, kommen Creatha und Alatep nach vorne
\end{quote}


\textbf{Creatha:}Ich fürchte, es hat keinen Sinne mehr. Es kann keinen Neuanfang geben mit so viel
Hass und diesen beiden zwischen den beiden Völkern

\textbf{Alatep:} Leider hast du Recht, ich pflichte dir bei. Es wird kein Zusammen für uns geben.

\textbf{Creatha:} Dann muß jedes Volk für sich alleine einen Neuanfang machen. Ich befehle die
Stürme und du die himmlischen Feuer. Lass uns die Schiffe trennen, das ist die einzige
Möglichkeit.

\begin{quote}
Stürme und Blitze brechen über die Schiffe ein. Tio und Furatha kämpfen weiter, während
Adran wieder auf die Bühne kommt und Kommandos gibt, das Schiff zu retten. Tior schlägt
Furatha nieder und will zum Todesstoß ausholen, aber Adran rennt dazwischen.
\end{quote}


\textbf{Tio:} Du willst mich aufhalten?

\textbf{Adran:} Nein, du wirst dich aufhalten! Du hast jetzt schon den Pakt der Wellen gebrochen,
willst du nun auch noch deine Ehre zu Grabe tragen? Der einzige, der Tiotep besiegen
kann, ist Tiotep selber. Tiotep siegt immer auch über seinen eigenen Zorn.

\begin{quote}
Adran geht ab brüllt weiter Kommandos und lässt Tiotep mit Furatha alleine
\end{quote}


\textbf{Tiotep:} Mein Herz willst du? Dann sollst du es haben. Ich siege über jeden, auch wenn es
Tiotep ist. Meine Leidenschaft, mein Hass, meine Liebe und mein Feuer. Das gebe ich
dir. Bist du nun zufrieden?

Furata nimmt das Herz an. Traurig aber immer noch von Hass verzehrt, dreht sich um und
geht ohne ein Wort von dannen.


\section{2. Akt 1.Szene:}

Die Caldrier landen in Engonien, ziehen weiter, siedeln und bauen sich ein neues Land auf.

\textbf{Caldrier 1:} Na endlich sind wir da, dem Flammenden sei gedankt, ich hätte es auf dem Schiff
keine Woche mehr ausgehalten.

\textbf{Caldrier 2:} Die paar Dranaer, die nach dem Sturm noch bei uns waren, haben uns wirklich
den Hals gerettet. Ohne sie wären wir nie angekommen.

\textbf{Caldrier 1:} Ich hoffe, die anderen haben den Sturm gut überstanden. Mir tut es mitlerweile
leid, was ich gesagt habe. Aber egal, daran können wir jetzt soweiso nichts ändern.
Na, wie wollen wir unser neues Land denn nennen?

\textbf{Adran:} Dieses Land hat schon einen Namen, guter Mann. Es heißt Andarra und Menschen
leben hier. Wir werden uns mit ihnen treffen und ihnen friedlich gegeüber treten, denn
meine Vorfahren kamen einst von hier. Wir werden weiterziehen müssen, bis wir eine
Stelle zum Siedeln finden.

\textbf{Alatep:} Nach Westen von hier aus, wenn die Einheimischen uns ziehen lassen.

\textbf{Adran:} Da bin ich sicher. Sie sind ein gutes Volk und selbst wenn nicht, sind sie unseren
Waffen und Kriegern nicht gewachsen.

\textbf{Tio:} Niemand wird unserem Volk ein Leid zufügen, dafür werde ich sorgen.

\textbf{Alatep:} Wieso so grimmig, Cousin? Freust du dich nicht auf den Kampf?

\textbf{Tio:} Wie sollte ich? Meine Freude ist mit der Wasserhexe verschwunden, aber meinem
Gelöbnis werde ich treu bleiben, das schwöre ich.

\section{2.Akt 2.Szene:}

Himmelsturm ist tot und die unsterblichen Dracors stehen an seinem Grab. Sie haben
gesehen, dass der Hydracorglauben mit ihm gestorben ist und sie sich keine Sorgen mehr zu
machen brauchen, dass die Caldrier glaubenstechnisch von rechten Weg abkommen. Nun
konzentrieren sie sich auf die Eroberung des Landes und den Ausbau des Reiches.

\textbf{Hydracor-Priester:} Ich bin der letzte Priester des Nachtblauen Drachen hier in diesem Land;
und auch wenn es noch Leute gibt, die einst aus Drana kahmen, so wird doch die Verehrung
des Nachtblauen mit mir heute sterben. Heute werden wir Adran Himmelsturm zur letzten
Ruhe betten und ihn bestatten, wie er es uns aufgetragen hat. Wenn ich ihn hinunter bringe
in sein Grab und es von innen schließe, wird er für immer weiter über euch wachen und dem
Volk, das er in dieses neue Land führte, Leitstern und Wegweiser sein. Pilgert zu seinem Grab
und ihr werdet gehört werden. Adran Himmelsturm hat uns alle gerettet. Uns und das Volk
von Drana, die wir auf der Überfahrt aus den Augen verloren haben. Wir alle schulden ihm
unser Leben. Prinzessin Illea hat ihm hier in Caldrien, unserem neuen Land, vier gesunde
Kinder geschenkt, genauso wie wir alle hier neue Familien gegründet haben. Er hat ein
starkes und gesundes Volk hinterlassen, das noch großes leisten wird. hiermit verabschiede
ich mich und weint nicht, denn er wird ewiglich bei euch sein.

\begin{quote}
Der Priester geht ab Alatep tritt auf, Tiotep in der Ecke
\end{quote}

\textbf{Alatep:} Nun, da der Navigator von uns gegangen ist und wir hierher gekommen sind, um ihn
zu ehren, ist es an der Zeit, uns zu offenbahren. Mein Name ist Alatep, Sohn des Justotep, des
Gottes der Gerechtigkeit, Sohn des Pyrdracor, des Ewig Flammenden. Caldrien, euer Land,
werdet ihr auf meinen Gesetzen aufbauen und meine Gerechtigkeit walten lassen. Ein jeder
soll sich dem Gesetz unterwerfen, auf dass wir eine große und starken Nation werden. Ihr
werdet Tempel bauen und ich werde die Finsternis vertreiben, auf dass die Dunkelheit keine
Macht über euch haben wird. Ruft mich, denn von heute an bin ich Alamar, euer Gott und
werde euch in ein goldenes Zeitalter führen. Wenn Feinde euch bedrohen wird Tiotep sie
zerschlagen mit all seiner Macht, denn er ist der Sohn des Desrutep, Herr des Krieges, Sohn
des ewig Flammenden. Er wird eure Feinde zerschmettern.

\section{2.Akt 3.Szene:}

Desrutep und sein neuer Sohn Gladius landen in Caldrien und treffen auf Tiotep. Dieser
regt sich natürlich darüber auf, dass sein toller Job nun von Papas neuem Liebling zunichte
gemacht werden soll. Daraufhin bietet er Papi an, dass er die Wölfe Nekas nimmt und sich
selber um die Sache kümmert. Widerwillig gewährt ihm Papi die Gunst und läßt ihm mal

\textbf{Gladius:} Wir sind schon seit einem Tag im Caldrischen Imperium. Wir sollten angreifen und
ihnen nicht mehr Zeit geben sich vorzubereiten, Vater.
Desrutep: Ich werde vorher mit deinem Bruder Tiotep sprechen. Er wird es nicht gerne sehen,
dass wir das Volk besiegen, das er hier aufgebaut hat. Außerdem sind die
Jahrhunderte, die wir uns nun nicht mehr gesehen haben, auch für einen Gott Zeit und
ich freue mich darauf, meinen Sohn wieder zu sehen.

\textbf{Gladius:} Da kommt er.

\textbf{Tiotep:} Vater! Was machst du hier. Warum ist das nekanische Heer nach Caldrien
gekommen?

\textbf{Desrutep:} Um das caldrische Imperium wieder in das nekanische Kaiserreich einzugliedern,
wohin es gehört, aber davon kann ich dir noch später berichten. Es freut mich sehr,
dich zu sehen und ich bin stolz auf das Volk, das du hier aufgebaut hast. Wer hätte
geglaubt, dass aus ein paar Flüchtlingen eine solche große Nation wird.

\textbf{Tio:} Sie sind wahrlich gut, Vater. Auch wenn ich den Ambitionen Alateps, oder Alamar, wie
er sich nun nennt, eine Nation zu erschaffen, nichts abgewinnen kann, so liebe ich
doch seine Krieger. So viele folgen mir mit meinem Namen auf den Lippen tapfer
in die Schlacht. Sie sind stark und ehrenvoll, schön und tödlich. Keines dieser
Barbarenvölker, die vorher hier lebten, konnten gegen sie bestehen.

\textbf{Desrutep:} Ja, das habe ich gesehen. Um so schwerer fällt es mir, sie vernichten zu müssen.

\textbf{Tiotep:} Du willst meine Krieger vernichten? Wiso?

\textbf{Derutep:} Es ist der Wille Pyrdracors, dass das caldrische Imperium erobert wird. Außerdem
sollen die überschüssigen nekanischen Truppen vernichtet werden. In Neka hat es 30
Jahre lang Krieg gegeben und nun stehen die kriegslüsternen Horden daheim dem
Wiederaufbau im Wege. Deswegen sollen sie vernichtet werden.

\textbf{Tio:} Truppen sterben lassen, ohne zu siegen, Vater! Wo liegt da die Ehre?
Derutep: Du hast recht. Viel Ehre liegt leider nicht in dieser Aufgabe. Aber wir sind Soldaten.
Deswegen wird sie auch dein Bruder Gladius ausführen und du wirst zurück nach
Neka gehen.

\textbf{Tio:} Was? Dieser unförmige Metallklotz? Er hat doch eine Registrierkasse, wo ein Kämpfer
sein Herz haben sollte. Der kämpft nur mit Zahlen und Statistiken auf einem Blatt
Papier. Wenn du deine Soldaten zu Tode langweilen willst, weil er nur Belagerungen
ausficht, weil sie wirtschaftlicher sind, ja dann bist du auf dem richtigen Weg.
Lass mich diesen Krieg hier führen, Vater. Ich werde gegen die Krieger in die
Schlacht ziehen, die mir in den letzten hundert Jahren folgten. Ich werde ihnen einen
ehrenvollen und glorreichen Tod geben. Beiden, ihnen und unseren Soldaten.

\textbf{Destrutep:} Ich brauche jemanden, der einen kühlen Kopf behält, Tiotep! Und das wird dein
Bruder Gladius sein. Du kannst ihm ja zur Seite stehen, wenn du unbedingt willst.

\textbf{Tio:} Dann gib mir diesen Kriegerorden, diese Wölfe Nekas und ich werde die Schlachten
schlagen. Deine tumbe Kriegsmaschine da kann dann immer noch hinter den Reihen
stehen und den General spielen.
Desrutep: So sei es dann. Aber denk daran, Gladius erfüllt den Auftrag des Ewig
Flammenden Pyrdracor. Deinem Bruder zu widersprechen heißt, dem Ewigen zu
wiedersprechen.

\section{2.Akt 4.Szene:}

Tiotep am Kartentisch. Er schlägt jede Menge Schlachten gegen die Caldrier und hat dabei
Spaß. Auftritt von Gladius. Ihm und Papa geht das zu langsam und er mischt sich ein.
Truppenverlegung und Angriff auf Burg Middenfelz

\textbf{Gladius:} Wir führen jetzt schon seit Jahren Krieg und immer sind noch keine Ergebnisse zu
sehen. Du bist zu waghalsig, zu ineffektiv.
Tiotep: Und du hast keine Ahnung, was es heißt, Krieg zu führen. Spürst du nicht die Glorie
der Schlacht? Das Feuer in den Augen der Menschen bei der Schlacht von Norngard,
oder der Eroberung von Salmar. Wir haben mit nur 150 Wölfen Nekas die Feste
erstürmt. Keine Belagerung, keine Katapulte. Mann gegen Mann und Schwert auf
Schild, das ist Krieg, das ist Glorie! Sieh nur meine Kinder, die Caldrier, wie sie
kämpfen. Nie habe ich würdigere Gegner gehabt. Ich bin so stolz auf meine Kinder.
Sie, das Fürstentum Middenfelz und die Fürstin ... Auch wenn sie Teils von diesem
Adran abstammt, so brennt in ihr doch ein solches Feuer, wie ich es noch nie in einem
Sterblichen gesehen habe.
\textbf{Gladius:} Gut, dass du davon sprichst. Dort werden wir als Nächstes angreifen. Und diesmal
befolgst du meine Befehle, Wort für Wort. Ich habe das Kommando und du tust, was
ich dir sage, Tiotep. Hier Angriff auf Burg Middenfelz JETZT!

Tiotep schaut Gladius haßerfüllt an, schnaubt und verlässt den Raum

\section{2.Akt 5.Szene:}

Creatha posiert an einem Ende der Bühne. Sie liest einen Brief vor.

\textbf{Creatha:} Meine liebe Freundin, hier gerät alles aus den Fugen… Mein Onkel und mein
Cousin Gladius sind gekommen und sie wollen unser Volk unterjochen. Nimmt das
denn nie ein Ende? Vergangene Nacht fiel Tiotep mit dem nekanischen Heer und den
Wölfen Nekas über die Burg Middenfelz her. Mit letzter Kraft gelang es der Fürstin
und ihren Verbündeten die anstürmenden Truppen abzuwehren. Der Nekanische
General forderte daraufhin eine Verhandlung am Ahnenfels… meinem Ahnenfels…


\begin{quote}


Fürstin und eine Caldrierin treten auf

Doch mein Cousin konnte noch nie gefallen an Verhandlungen finden …

Tiotep und ein roter Schwadroneur treten au

Tiotep mißachtete das Gesetz. Er zerstörte den Stein und enfesselte das Feuer. Es
wurde ein Massaker…

Schwadroneur jagt Caldrierin durch das Publikum, erwischt sie am Ende und schlägt sie

Großmütig verschonte er das Häufchen Elend, das das Feuer überstand und stellte der
Fürstin ein Ultimatum. Doch sie bot ihm mutig die Stirn.

Die Fürstin hilft dem Caldrier von der Bühne und Tiotep geht mit dem Schwadroneur
ebenfalls ab.

\end{quote}

Am heutigen Tage belagerte das nekanische Heer die Burg und besetzte ein
nahegelegenes Heiligtum deines Vaters. Sie waren im Begriff es zu schänden, als die
Fürstin und ihre Verbündeten eingriffen.

Ein nekanischer Priester steht an einem Becken. „Kaja“ stürmt über die Bühne. Wirft mit
Foki und Papiermessern um sich und verliert dabei einen Kelch. Der Priester sammelt ihn
ein. „Jerevan“ tritt auf, und greift den Priester an. „Jerevan“ wird niedergemacht und
gefangengenommen. „Jerevan“ wird Tiotep vorgeführt.

Jerevan, ein mutiger Kriegsherr, Verbündeter der Fürstin, wurde meinem Cousin vorgeführt.
Tiotep war auf eine Weise erzürnt darüber, dass sie es immer noch wagten sich ihm zu
widersetzen. Aber ebenso war er beeindruckt und erfreut darüber, dass die Fürstin und
ihre Verbündeten sich als so würdige Gegner erwiesen. So erhob er sein Glas…

\textbf{Tiotep:} Auf den Krieg und das Feuer der Leidenschaft!

\textbf{Jerevan:} Auf vergangene Schlachten! Auf die Ehre! Auf die glorreichen Siege der
Vergangenheit und ebenso auch auf die ehrenvollen Niederlagen. Auf jene, die auch
im Angesicht eines übermächtigen Feindes nicht ängstlich die Waffen strecken.
Sondern ihrem Gegner aufrecht und tapfer gegenüberstehen. So denn, auf die
hochverehrte Fürstin von Middenfelz!

\textbf{Tiotep:} Ihr beeindruckt mich, Fremder. Wohl gewählte Worte… Ihr seid euch bewust darüber
das ich euch hier an Ort und Stelle in Stücke reissen und die Überreste meinen Wölfen
zum Fraß vorwerfen könnte! Und dennoch zeigt ihr Mut und Ehre… So unterbreite
ich Euch ein Angebot. Ich fordere euch hiermit zur ehrenvollsten Art zu kämpfen.
Schwert gegen Schwert. Solltet ihr gewinnen, werde ich mich zurückziehen. Sollte der
Sieg jedoch mein sein, dann wird mir die Fürstin die Burg übergeben!

\textbf{Jerevan:} So sei es denn…

\textbf{Tiotep:} Ach, noch etwas. Richtet der Fürstin meinen ergebensten Gruß aus. Und es
würde mich freuen, wenn sie bei dem Kampfe zugegen ist.

\textbf{Creatha:} Was mein Cousin jedoch nicht ahnte war, dass eben diese Burg jener Ort war,
an dem Furatha das Herz des Löwenhäuptigen versteckte. Und während der tapfere
Fremde sich auf den vermutlich letzten Kampf vorbereitete, bargen seine Mitstreiter
dieses Kleinod und überbrachten es der Fürstin.
Was mein Cousin jedoch nicht ahnte war das wärend der fremde Kriegsherr sich
auf den vermutlich letzten Kampf vorbereitete, dessen Mitstreiter das Hertz des
Löwenhäuptigen bargen, dessen Reise eine eigende Legende wäre, und überbrachte es
der Fürstin.

\section{3. Akt 1. Szene:}

Fürstin, Jerevan und Tiotep treten auf. Tiotep und Jerevan duellieren sich. Mit einem
infernalen Donnern scheppert R2D2 auf die Bühne.

\textbf{Gladius:} Was soll dieses Spiel, Bruder?!
Gladius greift Jerevan an. Tiotep geht dazwischen. Duell zwischen Tio und Gladius . Tiotep
verliert sein Schwert. Flugrolle. Tio holt sein Schwert zurück. Gladius hackt Tiotep trotzdem

\textbf{Gladius:} Nun werde ich beenden, wozu du nicht im Stande warst. Fürstin, bereitet euch
darauf vor, dass das nekanische Heer eure Burg dem Erdboden gleichmachen wird!

Gladius zieht sich zurück. Die Fürstin und Jerevan kümmern sich um den verletzten Tiotep.
Die Fürstin pflanzt Tiotep das Herz wieder ein. Tiotep erwacht und sieht die Fürstin.

\section{3. Akt 2. Szene:}

Tiotep und die Fürstin gehen Hand in Hand über die Bühne. Als sie das andere Ende der
Bühne erreichen tritt Gladius mit einem Nekaner auf.

\textbf{Gladius:} Nun, mein Bruder, tritt zur Seite. Wir werden den Willen unseres Vaters
vollstrecken.

\textbf{Tiotep:} Unser Vater verlangt von seinem eigenen Sohn, dass er sich gegen seine Kinder
stellt… Nein, ich habe mich entschieden. Wenn ihr Caldrien unterwerfen wollt, dann müßt ihr
erst mich bezwingen. Ich werde es verteidigen, so wie ich es geschworen habe. Und immer
hätte tun sollen…

\textbf{Fürstin:} Nein du darfst dich nicht gegen deine Familie stellen.

\begin{quote}
Tiotep wirf ihr einen Blick zu und schiebt sie bestimt hinter sich                                                                 
 \end{quote} 

\textbf{Gladius:} *lacht* Dann soll es so sein, … Bruder!

\begin{quote}
Tiotep und Gladius duellieren sich erneut. Gladius hackt Tiotep wieder nieder. Kurz davor
wiederholt bezwungen zu werden bemächtigt er sich eines hinterhältigen Tricks um Gladius
zu blenden. Er schlägt ihn nieder. Als er ihm den Todesstoß versetzt, wirft sich die Fürstin
zwischen Gladius und Tiotep. Tiotep tötet beide mit einem Streich.
Tiotep bricht zusammen und realisiert erst jetzt wozu ihn sein Zorn getrieben hat. In Trauer
nimmt er die tote Fürstin auf und bettet sie ein Stück weiter entfernt nieder.
AUFTRITT DESTRUTEP
Destrutep drückt dem nekanischen Soldat seinen Helm in die Hand.                                                                
 \end{quote} 

\textbf{Destrutep:} Was hast du getan?! WAS HAST DU GETAN?! Er war dein Bruder!

\begin{quote}
Destrutep betrauert Gladius, während Tiotep um die Fürstin trauert. Destrutep schreitet zum
gebrochenen Tiotep hinüber.                                                              
 \end{quote} 


\textbf{Tiotep:} Vergib mir Vater, vergib mir…

\begin{quote}
Tiotep versucht das Gewand seines Vaters zu berühren. Dieser zieht es angewidert weg.                                                           
 \end{quote} 


\textbf{Destrutep:} Dir vergeben?! Du hast dich gegen dein eigen Fleisch und Blut gewandt. Dein
eigen Blut vergossen!!!

\begin{quote}
Destrutep geht zurück zu Gladius und zieht Ketten aus ihm.                                                        
 \end{quote} 


\textbf{Destrutep:} Für das was du getan hast, sollst du bis in alle Ewigkeit durch diese Ketten
gebunden werden, durch die Ketten DEINES STOLZES!

\textbf{Tiotep:} Vater… bitte…

\textbf{Destrutep:} Deinen Arm, ... Soldat! Als Tiotep nicht reagiert, wütend mit gebrochener Stimme
DEINEN ARM, SOLDAT!

\begin{quote}
Destrutep legt Tiotep in Ketten                                                      
 \end{quote} 


\textbf{Destrutep:} Du sollst auch fortan nicht mehr den Namen Tiotep tragen. TJOR sollst du
genannt werden... Eine Verstümmelung deines Namens, so wie du nur noch eine
Verstümmelung deiner selbst bist!
Am heutigen Tage soll das nekanische Volk Trauer tragen, denn heute habe ich
gleich 2 Söhne verloren...
An diesem schwarzen Tag wurde genug Blut vergossen. Das nekanische Heer
zieht sich zurück…

\begin{quote}
Destrutep wendet sich ab. Tiotep bleibt gebrochen zurück. Tiotep versucht die Fürstin zu
erreichen. Die Ketten halten ihn davon ab. Tiotep versucht die Ketten durchzubeissen. Er muß
aufgeben und windet sich in Schmerzen.

Szivar tritt auf. Er läßt sich ein wenig Zeit, er genießt den Anblick.                                                    
 \end{quote} 


\textbf{Szivar:} Das ist also übriggeblieben von dem einst so stolzen und edlen Tioptep... Nichts
weiter als ein Tier das seinesgleichen reißt... *lacht genüßlich*

\begin{quote}
Tiotep hat mit Szivars Auftritt aufgehört sich zu bewegen, schaut ihn aber nicht an. Er
verharrt mit gesenktem Kopf, in seiner Häufchen-Elend-Position. Als Szivar endet, springt er
mit einem Mal auf und geht auf Szivar los, wird aber von den Ketten zurückgehalten.                                                 
 \end{quote} 

\textbf{Tjor:} *fletscht ihn an* Bist du hergekommen, um mich zu verhöhnen?! Mich gebrochen zu
sehen, und dich an meinem Leid zu ergötzen, Szivar?! Tritt doch einen Schritt näher,
dann werde ich dich von DEINEM LEID erlösen. Ich werde dir dein Herz
herausreißen und damit dein lästerliches Maul stopfen, bevor ich dein häßliches
Gesicht zertrümmere!!!

\textbf{Szivar:} *ganz gelassen, immer noch mit einem höhnischen Lächeln* Aber, aber... Ich bin
hergekommen um dir zu vergeben, ... *genüßlich* mein lieber TJOR. Nach allem
was du mir und meinem Volk angetan hast! Einst standen wir auf verschiedenen
Seiten. Doch nun wurdest du verraten... Verraten von deinesgleichen! SIE haben dir
das angetan! Und dies macht und zu Brüdern!

\begin{quote}
Tjor spuckt aus                                              
 \end{quote} 


\textbf{Szivar:} Nicht DU bist der Verräter, du bist der VERRATENE! Dein Volk, DEIN EIGENER
VATER hat dich hintergangen und dich verstoßen. Bloß weil du geschützt hast, was
du liebtest. Sie hatten kein Recht das zu tun!

\begin{quote}
Szivar beobachtet Tjor, der nachzudenken scheint. Läßt ihn aber keinen Augenblick aus den Augen                                           
 \end{quote} 


\textbf{Szivar:} Wie ich bereits sagte, bin ich hergekommen um dir zu vergeben. Ich möchte dir ein
Angebot machen: Schließe dich uns an und nimm Rache an all jenen, die dir dieses
unsägliche Leid zugefügt haben! Nicht alle der deinen haben sich von dir abgewandt.
Deine Wölfe sind dir nach wie vor treu ergeben! Ich sprach bereits mit deinem
Cousin Alamar... Wir haben Frieden geschlossen und ein Bündnis geschmiedet. Es ist
auch sein Wunsch, dass du dich uns anschließt.

\begin{quote}
Szivar macht eine Pause und beobachtet Tjor                                           
 \end{quote} 

\textbf{Szivar:} Du hast hier ein Volk gegründet. Willst du zusehen, wie dein Vater dir auch noch
dieses nimmt?! Er nahm dir deine Ehre, er nahm dir deinen Namen und er nahm dir
deine Geliebte. Soll er dir nun auch noch deine Kinder nehmen?!

Szivar macht erneut eine Pause. Tjor schaut zu Boden und atmet heftig. Bei ihm rattern die
Mühlen in jeder Gehirnwindung.

\textbf{Szivar:} Hilf mir die Nekaner zu vertreiben. Sie werden einen hohen Blutzoll für das bezahlen,
was mit dir geschehen ist. Was sagst du, willigst du ein?

\begin{quote}
Tiotep denkt... Dann presst er mit haßerfüllten Blick hervor…                                         
 \end{quote} 

\textbf{Tjor:} Ich willige ein... Mögen alle jene Tjors Hass zu spüren bekommen, die es wagten sich
von ihm abzuwenden!

\textbf{Szivar:} *mit einem gleichsam zufriedenen und hinterlistigen Lächeln* Sehr gut... Sehr, sehr
gut...


\end{document}
