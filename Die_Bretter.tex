\documentclass[a5paper,7pt, twocolumn]{book}
\usepackage[ngerman]{babel} % Deutsch
\usepackage[utf8]{inputenc} % Dateiencodeing - wichtig
\usepackage{graphicx} % Bilder
%\usepackage{yfonts,color} %Initialien
%\usepackage[T1]{fontenc}
\usepackage{calligra}
\usepackage{amssymb}
%\newcommand*\initfamily{\usefont{U}{RoyalIn}{xl}{n}}
\pagestyle{headings}

\begin{document}

\setlength{\columnseprule}{0,1mm}



\chapter{Das Feld der Ehre}

\onecolumn[
Dies hier ist das Theaterstück zur Geschichte “Das Feld der Ehre”. Eine Lutz und Jerevan Geschichte adaptiert für die Bühnen Grenzbruecks, Condras und Engoniens vom ehrenwerten Theater Haberstaedt.
]

\pagebreak
\newpage

\section{Das Stück}


\textbf{Lutz}: Aber Herr, ich verstehe immer noch nicht warum wir in der kommenden Schlacht nicht auf Seiten der Tiburer kämpfen. Man sagt sie haben die besten Reiter überhaupt.

\textbf{Jerevan}: Weil diese ach so noblen Tiburer niemals einen dahergelaufenen fahrenden Ritter wie mich in ihren Reihen dulden würden, der kaum mehr besitzt als das, was er am Leibe trägt.

\textbf{Lutz}: Aber Herr eure Künste mit dem Schwert sind legendär und suchen landein, landaus ihres Gleichen.

\textbf{Jerevan}: In der Schlacht zählt das Können des Einzelnen kaum und wenn diese Ritter mit ihren Ahnentafeln, so lang, wie mein Arm nicht mit dir kämpfen wollen, weil dein Blut nicht so blau ist, wie ihres, dann ist die Schlacht schon für dich vorbei, bevor die Armeen das Feld der Ehre betreten haben.

\textbf{Lutz}: Aber Herr, jeder Ritter, der die Tugenden befolgt soll doch ein Bruder des Schwertes und in ihrer Gemeinschaft gleich sein.

\textbf{Jerevan}: Habe ich dir denn gar nichts beigebracht? Ein Ritter ist nichts anderes, als ein Kämpfer mit Waffe und Rüstung und der Absicht zu töten. Nicht anderes, als ein Mörder, der die ganzen Dinge, wie Ruhm, Ehre und die Gunstbezeugungen der Damen darüberstreut um vor anderen und sich selber zu verbergen, was er wirklich ist.

\textbf{Lutz}: Aber Ritter beschützen doch die Schwachen, wie diese Bauern dort.

\begin{quote}
  \textit{- Auftritt Margaret und ihre Zofe in zu sauberer Bauernkleidung, die ihre hochherrschaftlichen Gewänder nur notdürftig verdeckt. Die Zofe mit einem Korb am Arm.-
}
 \end{quote}

\textbf{Jerevan}: Zwei Bauersfrauen allein auf dem Weg zum Schlachtfeld? Habt ihr keine Angst, dass ihr überfallen werden könntet, … von einem Raubritter, oder sonst einem fiesen Schergen?

\textbf{Margaret}: Aber meine Tochter und ich sind doch nur zwei arme Bäuerinnen auf der Suche nach unserer Kuh. Wer sollte uns denn etwas antun wollen?

\textbf{Lutz}: Ihr braucht keine Angst zu haben. Herr Jerevan von Eibenhain ist ein ehrenwerter Ritter, der euch sicher beschützen wird bei eurer Aufgabe.

\textbf{Jerevan}: Wird er?

\textbf{Lutz}: Ja, die beiden Damen werden sicher dankbar sein für die Hilfe.

\textbf{Margaret}: Ja, sehr dankbar.

\textbf{Jerevan}: Ah! Es ist mir immer wieder eine Freude, wenn die Fundamente unserer Gesellschaft sich als dankbar erweisen wolle, … da sie meist in Naturalien bezahlen.
Ja, ich gelobe euch bei eurer Aufgabe zu helfen.

\textbf{Lutz}: Ja Herr, das sollten wir …

\textbf{Jerevan}: Ach Quatsch. Du wirst jetzt erst mal ein Lager machen und hier mit der Tochter Holz sammeln. Die gute Frau und ich gehen so lange mal ein Stück spazieren.

\textbf{Lutz}: Ja, aber Herr …

\textbf{Jerevan}: Nix “aber Herr”, tu was ich gesagt hab, oder du kriegst noch nen Satz hinter die Ohren.

\begin{quote}
  \textit{- Abgang Jerevan und Margaret - Lutz und Zofe bleiben alleine zurück und sammeln Holz. Die Zofe hebt sehr anmutig immer kleine Ästchen auf, die mal eigentlich nicht benutzen kann.
Bei der Abendvorstellung beugt sie sich immer mit dem verlängerten Rücken zum Publikum.}
 \end{quote}

\textbf{Lutz}: Bauern, ja?

\textbf{Zofe}: Ja, wir sind zwei einfache Bäuerinnen auf der Suche nach unserer Kuh.

\textbf{Lutz}: Ja, ähem, natürlich, aber ihr solltet wirklich aufpassen ein Schlachtfeld ist kein Platz für allein reinsende … Bauersfrauen. Es ist gefährlich.

\textbf{Zofe}: Auch wenn dies hier ein Schlachtfeld ist, so gibt es doch viele edle Herren hier, die der hohen .. meiner Mutter nichts antun würden.

\textbf{Lutz}: Bei meinem Herren Jerevan braucht ihr euch nicht zu fürchten. Es mag ungehobelt und schroff sein, aber in seinem Kern ist er ein guter Mann und sein Wort würde er niemals brechen.
Andere Ritter wären da sicher nicht so. Ich fürchte der Krieg bringt das Schlechteste in den Menschen hervor und es gäbe sicher einige Ritter, die eure Zwangslage ausnutzen würden zu ihrem persönlichen Vorteil.

\textbf{Zofe}: Doch gerade in diesen wiedrigen Zeiten kommen der wahre Edelmut und die Tugenden zum Vorschein, der edles Blut vom tumben Volk unterscheidet…

\textbf{Lutz}: Das edle Blut hat noch niemanden davor bewahrt Menschen auszunutzen. Im Gegenteil, die schlimmsten Dinge wurden im Namen angeblich edlen Blutes getan.
Die Tugenden der Ritter sind es, die ihn erheben und zu etwas besserem machen, nicht sein Blut.

\textbf{Zofe}: Die einzige Tugend, die jene von uns, die nicht von edlem Stand sind, beachten sollten ist die Demut.

\begin{quote}
  \textit{(weiter Stöcke sammelnd)}
 \end{quote}


\textbf{Lutz}: Was außer Demut noch wichtig ist, das ist die Wahrheit. Ich kann euch helfen, wirklich, wenn ihr mir die Wahrheit sagt.

\textbf{Zofe}: Schwöre zunächst, dass du nichts von dem, was ich dir jetzt sagen werde weiter erzählst.

\textbf{Lutz}: Ich schwöre es.

\textbf{Zofe}: Ich diene ihrer Hoheit Margarethe von Hohenstaaden.

\textbf{Lutz}: Was?!

\textbf{Zofe}: Ich habe wirklich alles versucht, aber die hohe Herrin ließ sich durch nichts umstimmen; sie will diese Aufgabe unbedingt selbst in die Hand nehmen.

\textbf{Lutz}: Was für eine Aufgabe?

\textbf{Zofe}: Das Land leidet unter diesem Krieg. Die Bauern flehen Myrn um Frieden an. Die hohe Herrin hat sich aufgemacht bei ihrem Herren Gemahl, vor den Augen aller seiner Edlen, Frieden zu erbitten. Er wird es ihr nichts abschlagen können, nicht hier.

\textbf{Lutz}: Ich werde alles tun, um dir zu helfen. Aber leise jetzt, hier kommen mein Herr und deine Herrin.

\begin{quote}
  \textit{- Auftritt Jerevan und Magdalena -}
 \end{quote}


\textbf{Jerevan}: Was? Das soll ein Lager sein?

\textbf{Lutz}: Ah, Herr … Ähem, seid ihr weit gekommen?

\begin{quote}
  \textit{- zieht Margaret vom Ritter weg und positioniert sich zwischen ihnen -}
 \end{quote}

\textbf{Jerevan}: Hä? … hm … heute werden wir nicht weiter gehen. Auf dem Schlachtfeld liegen nur noch die Überbleibsel des letzten Tages und die Heere haben sich noch nicht aufgestellt. Wir werden morgen, bevor es hell wird, zu den Hohenstaadenern gehen.
Lutz, Wein!

\textbf{Lutz}: Das geht nicht Herr, den habt ihr gestern aufgetrunken.

\begin{quote}
  \textit{- Auftritt Bastard -}
 \end{quote}



\textbf{Bastard}: Dann trifft es sich ja gut, dass ich noch welchen habe.

\begin{quote}
  \textit{- Jerevan springt auf, geift ans Schwert, zieht aber noch nicht -}
 \end{quote}

\textbf{Jerevan}: Bleib stehen, wer bist du?

\textbf{Bastard}: Der Bastard von Malmais, ein müder Ritter, der ein warmes Feuer gegen einen Becher Wein tauscht, wenn ihr auf der richtigen Seite steht.
Tibur oder Hohenstaaden?

\textbf{Jerevan}: Hohenstaaden, ab dem morgigen Tage.


\begin{quote}
  \textit{- Bastard steckt sein Schwert weg -}
 \end{quote}


\textbf{Bastard}: Dann werden wir auf der gleichen Seite stehen.

\textbf{Jerevan}: Setzt euch und füll die Becher, Lutz!
Gibt es gutes Plündergut zu holen bei den Tiburern?

\textbf{Bastard}: Es geht. Die neun Jahre haben das Land ausgezehrt und jeder nimmt das, was er kriegen kann. Zwei so hübsche unbeschädigte Bäuerinnen wie ihr sie habt, findet man nur noch selten.

\textbf{Jerevan}: Das einzige, was ich  auf dem Weg hier her gefunden habe. Schon ein wenig armselig.

\textbf{Magdalena}: Würdet ihr mich auf das Schlachtfeld begleiten?

\textbf{Jerevan}: Wiso auf das Schlachtfeld?

\textbf{Magdalena}: Ich muss meine Wäsche aufhängen.

\textbf{Jereva}: An deine Wäsche gehen wir dir noch früh genug.

\textbf{Bastard}: An diese Wäsche würde ich dir auch gerne mal gehn.

\textbf{Jerevan}: Finger, weg! Das ist meine Wäsche.

\textbf{Bastard}: Aber ein Ritterbruder wird doch wohl seine Beute teilen.

\textbf{Jerevan}: Na mit wie vielen Ritter hast du nach dem letzten Kampf dein Plündergut geteilt.

\textbf{Bastard}: Na mit mir. Ha ha ha

\textbf{Magdalena}: Aber ich muss dringend auf das Schlachtfeld, die Saat ausbringen.

\textbf{Bastard}: Etwas Samen kann ich sicherlich da beisteuern Holdeste.

\textbf{Jerevan}: Meine Furche mein Samen. Such dir selber was zum beackern.

\textbf{Magdalena}: Aber ich muss noch meine Kuh finden und melken.

\textbf{Bastard}: Mit Eutern und Zizen kenne ich mich prächtig aus. Da hat sich noch keine Kuh beschwert.

\begin{quote}
  \textit{- Jerevan seuftst entnervt, lehr den Rest vom Becher in einem Zug und packt ihn weg -}
 \end{quote}


\textbf{Jerevan}:
zu Magdalena - Na wenns unbedingt jetzt sein muss.
zum Bastard - Meine Kuh, meine Milch. Machs dir hier ruhig bequem, ich bin bald wieder zurück.


\begin{quote}
  \textit{- Abgang Jerevan und Magdalena. Derweil sind Lutz und die Zofe anscheinend eingeschlafen und der Bastard wähnt sich unbeobachtet -}
 \end{quote}


\textbf{Bastard}: Wenn das zwei Bauersfrauen sind, dann bin ich der neue König von Grenzbrueck.
Na vielleicht finde ich in ihrem Korb etwas, dass ein wenig Licht in die Sache bringt.
Ah … sieh da ein Siegelring mit dem Wappen von … Hohenstaaden. Die Frau des Großherzogs alleine auf dem Schlachtfeld. Nur beschützt von einem zerlumpten, fahrenden Ritter und seinem Knappen. Ha, wenn sich da kein Vorteil draus schlagen lässt.

\begin{quote}
  \textit{- Abgang Bastard Abgang Bühne links mit dem Ring -}
 \end{quote}


\textbf{Zofe}: Nein, er hat es bemerkt, was sollen wir nur tun?

\textbf{Lutz}: Wir werden ihm erst einmal folgen und sehen, wo er hingeht. Dann werden wir es meinem Herren sagen.

\begin{quote}
  \textit{- Abgang Lutz und Zofe. Abgang Bühne links hinter dem Bastard her -}
 \end{quote}

- hier evtl. Pause einfügen. Ist allerdings meist nicht nötig, da das Stück kurz genug ist um auch ohne Pause gespielt werden zu können. -

\begin{quote}
  \textit{- Auftritt Bühne rechts Tessa von Ravur -}
 \end{quote}


\textbf{Tessa von Ravur}: Dieser Krieg zehrt das Land auf. Kein Dorf, das nicht geplündert, keine Stadt unberührt und das alles ob unserer menschlichen Begierden. Wieviele treue und tapfre Diener muss ich noch in den Tod schicken, bis ihr Blut und das unsre, Ehre und Pflicht genüge getan hat. Ihr Ewgen, wieviel Leid muss unser Volk noch erdulden bevor es für unsre Taten gebüßt hat?

\begin{quote}
  \textit{- Auftritt Bühne links Bastard Lutz und Zofe. Lutz und Zofe schleichen hinter dem Bastard her und verstecken sich vor den beiden -}
 \end{quote}

\textbf{Bastard}: Ah meine geliebte Mutter, ich habe habe euch gesucht, an diesem schönen Abend.

\textbf{Tessa von Ravur}: Was willst du hier, Bastard?

\textbf{Bastard}: Das bin ich, meine Dame. Aber habe ich nicht im Namen meines Vaters, eures verstorbenen Ehemannes, ein wärmeres Willkommen verdient?

\textbf{Tessa von Ravur}: Mein Mann war zehn mal, der Mann der ihr seid, Bastard von Malmaise.

\textbf{Bastard}: Und nun ist er zehnmal so tot und ich bin noch hier, um euer blutiges Tagewerk zu tun.

\textbf{Tessa von Ravur}: Für solch Taten seid ihr gerade gut genug. Sagt was ihr wollt oder schert euch von dannen.

\textbf{Bastard}: Ich will das was mir zusteht, mein Erbe, das Erbe meines Vaters.

\textbf{Tessa von Ravur}: Euch steht nichts zu Bastard! Auch wenn mein Mann und alle meine Kinder tot in diesem Krieg gefallen sind, so werdet ihr niemals erben, was nicht euer ist. Was in Acrulons Namen lässt euch denken, dass ich meine Meinung ändern werde.

\textbf{Bastard}: Weil ich euch bringe, was sich euer Herz am meisten wünscht, Friede!

\textbf{Tessa von Ravur}: Friede! Aus eurer blutgen Hand. Ihr seid noch mehr von Sinnen als sonst, wenn ihr glaub das erreichen zu können, was neun Jahre Krieg nicht geschafft haben.

\textbf{Bastard}: Ich sehe heute viel klarer als jemals zuvor. Versprecht mir mein Erbe und ich werde euch den Frieden bringen.

\textbf{Tessa von Ravur}: Sag wie du das bewerkstelligen willst oder tritt aus meinen Augen.

\textbf{Bastard}: Gebt mir erst euer Versprechen.

\textbf{Tessa von Ravur}: Für den Frieden. Ja … für den Frieden würde ich sogar euch als Erben anerkennen, aber wenn ihr versucht mich zum Narren zu halten, wirds dir schlecht ergehen, auch das verspreche ich dir Bastard.

\textbf{Bastard}: Dazu werdet ihr keinen Grund haben … Mutter, - denn ich habe mit eigenen Augen gesehen, wie Margaret von Hohenstaaden alleine nur von einem einzigen fahrenden Ritter bewacht auf dem Schlachtfeld unterwegs ist. Wenn wir sie gefangen nehmen können wir ihren Mann dazu bringen aufzugeben, den Tiburern ein Stück Land abtreten und der Krieg ist vorbei.

\textbf{Tessa von Ravur}: Ihr lügt. Sie würde nie alleine hier in der Schlacht sein.

\textbf{Bastard}: Ihr seid schnell mit eurem Urteil bei der Hand … Herrin. Denn wenn sie nicht hier wäre, woher habe ich dann das hier? (Zeigt ihr den Siegelring)

\begin{quote}
  \textit{- Tessa von Ravur Sprich nun zum Publikum und der Bastard steht hinter ihrer Schulter und flüstert ihr ins Ohr. Es soll klar für das Publikum erkennbar sein, dass dies eigentlich ein innerer Dialog von Tessa von Ravur ist -
}
 \end{quote}

\textbf{Tessa von Ravur}: Ein Siegelring … mit dem Siegel von Hohenstaaden? Dann ist es wirklich war. Aber das was ihr vorschlagt ist Hochverrat. Hand an die Frau des Großherzogs, meines Lehnsherren zu legen … und sei der Grund auch noch so nobel. Wie kann aus schlechter und schändlicher Tat Frieden folgen? Meine Ehre, meine Pflicht verbieten es … ich kann die Ehre meines Blutes nicht so beschmutzen.

\textbf{Bastard}: Ihr meint die Ehre des Blutes eures Gemahls, eurer Kinder und eures Volkes, das der Großherzog Eimerweise verschüttet hat, für Gründe die schon seit Jahren null und nichtig sind. Er, der immer noch weiterkämpft, obwohl alles für das wir einst am Anfang das Schwert erhoben haben, sowieso nicht mehr zu retten ist. Er, der diesen sinnlosen Krieg immer noch weiterführt, weil er zu stolz ist zuzugeben, dass man ihn nicht mehr gewinnen kann, selbst wenn wir die Tiburer besiegen und er, der dafür jeden von uns opfern würde, dem seid ihr noch verpflichtet?

\textbf{Tessa von Ravur}: Ja dem bin ich verpflichtet mit meiner Ehre bis zum letzten Atemzug. Meine Hand gegen sein Blut erheben, wo ich doch bei den Ewgen geschworen habe ihm treu zu sein? Undenkbar!

\textbf{Bastard}: Ihr seid eurem Volk verpflichtet, eurer Familie, eurem Blute. Denen, die Gefallen sind und denen, die noch leben, aber sicher auch sterben werden, wenn dieser Wahnsinn weiter geht. Gebt mir 50 Ritter und noch vor dem Morgengrauen wird Margaret von Hohenstaaden in euren Händen und der Frieden gewonnen sein.

\textbf{Tessa von Ravur}: Neun Jahre des Krieges. Jahre in denen unsere Felder über und über getränkt werden mit unserem Blut. Wenn die Morgenröte mich zwingt wieder auf ein Schlachtfeld zu treten und mehr gute Männer, Söhne, zu opfern. Die Schreie der Trauernden und das Stöhnen der Sterbenden sind zu unseren Schlafliedern geworden. Um des Friedens willen… werde ich alles Verraten um dieses Leid zu beenden. Ja, Bastard eure Wünsche sollen euch erfüllt werden. Ihr werdet 50 Mann bekommen und bringt mir Margaret von Hohenstaaden!

\begin{quote}
  \textit{- Die Zofe zerbricht einen Stock. Tessa von Ravur und der Bastard werden aufmerksam -}
 \end{quote}


\textbf{Zofe}: Nein, sie wollen meine Herrin gefangen nehmen.

\textbf{Lutz}: Pscchhht, leise.

\textbf{Bastard}: Was?

\textbf{Tessa von Ravur}: Wer ist da? Wer lauert da im dunklen Gestrüpp.

\textbf{Bastard}: Das ist dieser verdammte Knappe mit der Bauersmagt.

\textbf{Tessa von Ravur}: Sie dürfen davon nichts berichten. Töte sie! Ich komme mit den Rittern nach.

\textbf{Bastard}: Das mache ich, bleibt stehen ihr Würmer!

\begin{quote}
  \textit{- Es beginnt eine Verfolgungsjagt. Der Bastard verfolgt die Zofe und Lutz. Diese Verfolgungsjagt soll lustig gestaltet werden, damit die Zuschauer zwischen den beiden anspruchsvollen Szenen auch etwas zu Lachen bekommen.
Diese Szene muss stark an die örtlichen Gegebenheiten angepasst werden und endet wie folgt: -}
 \end{quote}


\textbf{Lutz}: Wir können vor ihm nicht ewig weglaufen, wir müssen eine Falle stellen.

\textbf{Zofe}: Aber wie denn nur?

\textbf{Lutz}: Was hast du dabei?

\textbf{Zofe}: Nur was in den Korb ist, diesen Stickrahmen.

\textbf{Lutz}: Ich hab eine Seil und das Tuch zum polieren der Rüstung. Warte, damit können wir ihm eine Falle stellen.

\begin{quote}
  \textit{Dann verstecken sie sich hinter zwei “Bäumen” und spannen das Seil. Der Bastard tapt herein und fällt. Lutz wirft ihm das Poliertuch über den Kopf und die Zofe stülpt ihm den Stickrahmen drüber, damit es fest bleibt. Während der Bastard blind herumtorkelt wickeln die beiden ihn in das Seil ein, fesseln ihn somit kurzzeitig und ergreifen die Flucht.
Der Bastard flucht laut, tölpelt etwas herum, entledigt sich schließlich der Gegenstände und geht dann fluchend von der Bühne ab.
}
 \end{quote}

\textbf{Bastard}: Ich kriege euch ihr Maden.


\begin{quote}
  \textit{- Abgang Bastard rechts -
}
 \end{quote}

\begin{quote}
  \textit{- Auftritt Tiburer mit Leiche (Puppe) im Arm von links. Er legt die Leiche zentral auf der Bühne direkt vor der Rückwand ab, torkelt etwas und geht auf halbem Weg zwischen Puppe und Aufgang rechts im Frontbereich zum Publikum zu Boden. -
}
 \end{quote}

\begin{quote}
  \textit{- Aufgang Jerevan und Margaret von Rechts Arm in Arm -
}
 \end{quote}

\textbf{Jerevan}: Guck mal da, hinter diesem dunklen dichten Gebüsch … da ist sicher deine Kuh.

\textbf{Margaret}: Seht da, ein verwundeter Ritter.

\textbf{Jerevan}: Ach, nicht noch einer. Jetzt sag nicht, um den willst du dich auch noch kümmern. Die Euter deiner Kuh müssen doch schon fast am bersten sein.

\textbf{Margaret}: Ich muss mich doch um ihn kümmern.

\textbf{Jerevan}: Nein, das musst du nicht. Das ist eh ein Tiburer, wenn ich ihm morgen auf dem Feld begegne, dann werde ich ihn sowieso töten. Dass kann er auch jetzt haben...

\begin{quote}
  \textit{- Jerevan zieht sein Schwert -
}
 \end{quote}


\textbf{Jerevan}: Schneller und Schmerzloser.

\textbf{Margaret}: Nein!

\textbf{Jerevan}: Ach, was für eine dämliche Idee das gewesen ist euch beide mitzunehmen. Wir laufen jetzt schon die ganze Nacht hier herum und du hast nichts besseres zu tun, als dich um diese elenden Ritter zu kümmern, die nicht gut genug waren um auf sich aufzupassen und sowieso bald vor dem Fährmann stehen werden. Ach verdammte Scheiße …

\textbf{Tiburer}: Vielen Dank für eure Hilfe meine Dame. Ich verdanke euch mein Leben, ich stehe tief in eurer Schuld, bei meiner Ehre, wenn ich etwas …

\textbf{Jerevan}: Verschwinde du Idiot und sei froh, dass du noch lebst. Ich muss noch eine Kuh finden.

\begin{quote}
  \textit{- Jerevan tritt den Tiburer. Abgang Tiburer Bühne rechts -
}
 \end{quote}

\begin{quote}
  \textit{- Auftritt Bühne links Lutz und Zofe -
}
 \end{quote}

\textbf{Jerevan}: Halt wer … Ach du bist es. Unangekündigt in mein Schwert zu laufen ist ein guter Weg …

\textbf{Lutz}: Herrin, Herrin, sie kommen um euch zu holen, sie haben es herausgefunden und wollen euch gefangen nehmen um …

\textbf{Jerevan}: Das heißt Herr du Idiot und wer kommt um mich zu holen und was redest du überhaupt?

\textbf{Lutz}: Herrin, Großherzogin von Hohenstaaden, die haben herausgefunden, wer ihr seid und wollen euch entführen, um euren Mann den Großherzog von Hohenstaaden zu erpressen.

\textbf{Jerevan}: Ach, verdammte Scheiße.

\begin{quote}
  \textit{- Auftritt Bastard Bühne links, die Zofe läuft zu ihrer Herrin und betüddelt sie -
}
 \end{quote}

\textbf{Bastard}: Ja genau.

\textbf{Jerevan}: Bleib genau da stehen, wenn du weiterleben willst.

\textbf{Bastard}: Was willst du denn? Sag, was willst du für die Frau, die dich die ganze Zeit zum Narren gehalten hat?

\textbf{Jerevan}: Bäuerin oder nicht, ich hab ihr mein Wort gegeben.

\begin{quote}
  \textit{- Der Bastard wirft ihm einen Geldbeutel zu, der an Jerevans Brust abprallt und zu Boden fällt -
}
 \end{quote}

\textbf{Bastard}: Ich bezahle für dein Wort, mehr als ein fahrender Ritter jemals hoffen kann zu besitzen.

\textbf{Jerevan}: Mein Wort kann man mit keinem Gold der Welt kaufen …

\textbf{Bastard}: … aber mit Stahl erschlagen.

\begin{quote}
  \textit{- Bastard und Jerevan kämpfen. Jerevan besiegt den Bastard. Hier ist eventuell Platz für Rüstungs oder Schmiedeangebote -
}
 \end{quote}

\textbf{Jerevan}: Pah, Idiot. Und nun zu uns.

\begin{quote}
  \textit{- Auftritt Bühne links Tessa von Ravur mit Rittern -
}
 \end{quote}

\textbf{Tessa von Ravur}: Was auch immer er gewesen war … Ritter. Er war von meinem Blute und du hast ihn erschlagen. Dafür wirst du sterben.

\textbf{Jerevan}: Jerevan von Eibenhain gegen 50 Ritter? Na das ist doch mal ein Kampf von dem es sich lohnen wird zu berichten.

\begin{quote}
  \textit{- Margaret entdeckt die Puppe -
}
 \end{quote}

\textbf{Margaret}: Neiiiin, mein Mann … tot danieder!

\textbf{Tessa von Ravur}: Verdammt, damit ist der Frieden dahin und der Morgen wird das Schlachtfeld erneut rot färben.
Vor meinem toten Lehnsherren werd ich dich nicht erschlagen … Ritter, aber wenn die Sonne aufgeht, dann bist du tot. Ihr da! Nehmt ihn mit.

\textbf{Tessa von Ravur}: Verdammt, damit ist unsere Hoffnung dahin und die anderen Feldherren werde sich nicht zu einem Frieden umstimmen lassen. So wird auch dieser Morgen das Schlachtfeld erneut rot färben. Vor meinem toten Lehnsherren werd ich dich nicht erschlagen, doch wisse so denn die Schlacht beginnt, so wird es keine Gnade geben. Dann bist du tot!

\begin{quote}
  \textit{- wirft den Siegelring auf Jerevan, der von seiner Brust abprallt -
}
 \end{quote}

Bis dahin Ritter … Jerevan von Eibenhain. - Ihr da! Nehmt ihn mit.

\begin{quote}
  \textit{- Abgang links Tessa von Ravur mit Rittern und Leiche vom Bastrad -
}
 \end{quote}


\textbf{Jerevan}: Ach scheiße.

\textbf{Lutz}: Herr? Was sollen wir nun machen?

\textbf{Jerevan}: Hm, wenn wir schnell handeln, dann können wir uns vielleicht durch die Tiburer Linien schleichen und irgendwie wegkommen.

\textbf{Lutz}: Herr, aber ihr sagtet doch, dass die TIburer uns niemals aufnehmen werden.

\textbf{Jerevan}: Ja, das werden sie auch nicht, wenn sie uns erwischen, dann sind wir tot. Sie sind bessere Reiter und und kennen das Land. Unsere Überlebenschancen gibt es nicht wirklich.

\textbf{Lutz}: Wenn wir nicht überleben können, vielleicht sollten wir dann hier bleiben.

\textbf{Jerevan}: Hast du sie nicht gehört? Hier sind wir garantiert tot, in Tibur haben wir vielleicht noch eine Möglichkeit.

\textbf{Lutz}: Aber ihr habt ihr doch euer Wort gegeben.

\textbf{Jerevan}: Ich wollte die Kuh einer Bäuerin suchen und sie ist nur die ganze Zeit übers Schlachtfeld gelaufen und hat Verwundete geheilt, von beiden Seiten … dumme Kuh … dankbar erweisen … wenn ich nicht lache.

\textbf{Lutz}: Ihr habt gelobt ihr bei ihrer Aufgabe zu helfen und sie wollte Frieden für das Land.

\textbf{Jerevan}: Und wie soll ich das erreichen?

\textbf{Lutz}: Die Ritter denen sie geholfen hat, die waren doch bestimmt dankbar, oder?

\textbf{Jerevan}: Ja klar, die noblen aufgepufften Edelmänner haben bei ihrer Väter Väter Ehre geschworen Ihre Schuld … blablabla

\textbf{Lutz}: Wie wärs, wenn wir zu diesen Ritter …

\textbf{Jerevan}: Ja, vielleicht könnte uns einer von denen durch die feindlichen Linien …

\textbf{Lutz}: Herr …

\textbf{Jerevan}: Halts Maul, ich muss nachdenken. … Du Lutz ich habe eine Idee. Als wir übers Schlachtfeld sind hat die Herzogin Ritter von beiden Seiten geheilt, die ihr jetzt ihr Leben schulden … ich hab eine Idee … komm mit.

\textbf{Lutz}: Ja, wunderbar Herr.

\begin{quote}
  \textit{- Abgang Bühne links Jerevan und Lutz -
-Margareth und die Zofe bleiben auf der Bühne, Margareth hinter dem Umhang der Zofe, die mit dem Rücken zum Publikum steht. Sie zieht eine weiße Perücke auf und bleibt hinter dem Umhang versteckt.-
- Auftritt Bühne rechts das Heer von Tibur, Auftritt Bühne links das Heer von Hohenstaaden -
}
 \end{quote}

\textbf{Teoderederederich}: Und so treffen die Heere erneut aufeinander. Reite Tibur reite. Eomund du meine beste Lanze, du wirst den ersten Stoß führen.

\textbf{Eomund}: Mein Fürst Teoderederederich ich kann nicht. Bei meiner Väter Väter Väter Väter Ehre Karren habe ich geschwohren im heutigen Kampf weder Lanze noch Schwert zu erheben gegen unseren Feind.

\textbf{Teoderederederich}: Eodmund mein treuer Reiter. Du wars immer treu und so werde ich nichts verlangen, was wider deine Ehre ist.

\textbf{Tessa von Ravur}: Dann wirst du Gisber ehrenvollster Ritter Hohenstaadens die Schlacht beginnen. Spanne die Armbrust und schleiche dich um ihre Flanke.

\textbf{Gisber}: Herrin auch ich habe geschworen in diesem Kampfe nicht die Armbrust gegen den Feind zu erheben.

\textbf{Teoderederederich}: Dann wirst du Helmhammerhaar den Angriff führen.

\textbf{Helmhammerhaar}: Mein Fürst auch ich habe geschworen.

\textbf{Tessa von Ravur}: Dann du Juliette.

\textbf{Juliette}: Herrin auch ich habe geschworen die Armbrust ruhen zu lassen.

\textbf{Teoderederederich}: Dann du Erzsteintrutzheim.

\textbf{Erzsteintrutzheim}: Auch ich, mein Fürst.

\textbf{Tessa von Ravur}: Du auch Fleur?

\textbf{Fleur}: Auch ich.

\textbf{Tessa von Ravur und Teoderederederich}: Wem habt ihr das geschworen?

\textbf{Alle}: Ihr haben wir das geschworen.

\begin{quote}
  \textit{Margaret hat sich derweil unter dem Umhang ihrer Zofe umgezogen mit der Perücke mit den weißen Haaren und kommt nun als göttlich gesanndte heilige Margaret zentral zwischen den beiden Parteeien hervor.}
 \end{quote}


\textbf{Tessa von Ravur und Teoderederederich}: Ein Wunder.

\textbf{Margaret}: Das Land hat genug geblutet. Ob gerechter Streit oder noble Sache, für die eine Seite, oder die Andere sei gleich. Barmherzigkeit und Mildtätigkeit ist Tugend eines jeden Ritters und 9 Jahre des Blutvergießens sind genug. Steckt die Schwerter ein, betrauert die Toten und Ehrt ihr Opfer. Geht nach Hause zurück zu euren Familien, baut das Land wieder auf und seid Barmherzig zu euren einstigen Feinden.

\textbf{Alle}: Es sei Frieden.

\begin{quote}
  \textit{- Abgang Alle in einer Kolonne durch die Mitte -
}
 \end{quote}

\end{document}
